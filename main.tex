\documentclass[12pt,openright,twoside,a4paper]{article}
\usepackage[brazilian]{babel}

% ---
% Pacotes básicos 
% ---
\usepackage[top=20mm, bottom=20mm, left=20mm, right=20mm]{geometry}
\usepackage[T1]{fontenc}		% Selecao de codigos de fonte.
%\usepackage{notomath}
%\usepackage{arev}
\usepackage{lmodern}			% Usa a fonte Latin Modern
%\usepackage{cmbright}
%\usepackage[OT1]{fontenc}
%\usepackage[utf8]{inputenc}		% Codificacao do documento (conversão automática dos acentos)
\usepackage{transparent}
\usepackage{amsfonts, amsmath, amssymb, amsthm}
\usepackage{cancel}
\usepackage{lastpage}			% Usado pela Ficha catalográfica
\usepackage{indentfirst}		% Indenta o primeiro parágrafo de cada seção.
\usepackage[skip=10pt plus1pt, indent=40pt]{parskip}
\usepackage{color}				% Controle das cores
\usepackage{xcolor}
\usepackage{graphicx}			% Inclusão de gráficos
\usepackage{microtype} 			% para melhorias de justificação
\usepackage{tikz, tabularx, pgfplots} 
\usepackage{geometry}
\usepackage{svg}
\usepackage{multicol}
\usepackage{caption, subcaption}
\usepackage{csquotes}           %
\usepackage[colorlinks=true,linkcolor=black,anchorcolor=black,citecolor=black,filecolor=black,menucolor=black,runcolor=black,urlcolor=black]{hyperref}%
\usepackage{ragged2e}%
\usepackage{framed}
\usepackage{listings}           % para inclusão de código
\usepackage{booktabs}
\usepackage[backend=bibtex]{biblatex}           %coloque backref=true se quiser que fale em qual pagina foi citado
\usepackage{tcolorbox}
\usetikzlibrary{calc}
\usepackage{fancyhdr}
\fancyhf{}
\renewcommand{\headrulewidth}{0pt}
\fancypagestyle{plain}{
	\fancyfoot[R]{\thepage}}
\pagestyle{plain}
\addbibresource{bibliografia.bib} % O arquivo de bibliografia

\definecolor{shadecolor}{rgb}{0.9,0.9,0.9}

\theoremstyle{definition}
\newtheorem{definition}{Definição}[section]
\newtheorem{example}{Exemplo}[section]
\newtheorem{practice}{Prática e solução}[section]
\renewcommand{\sin}{\operatorname{sen}} %resolvendo um probleminha com \sin


\definecolor{codegreen}{rgb}{0,0.6,0}
\definecolor{codegray}{rgb}{0.5,0.5,0.5}
\definecolor{codepurple}{rgb}{0.58,0,0.82}
\definecolor{backcolour}{rgb}{0.95,0.95,0.92}

\lstdefinestyle{mystyle}{
	backgroundcolor=\color{backcolour},   
	commentstyle=\color{codegreen},
	keywordstyle=\color{magenta},
	numberstyle=\tiny\color{codegray},
	stringstyle=\color{codepurple},
	basicstyle=\ttfamily\footnotesize,
	breakatwhitespace=false,         
	breaklines=false,                 
	captionpos=b,                    
	keepspaces=true,                 
	numbers=left,                    
	numbersep=5pt,                  
	showspaces=false,                
	showstringspaces=false,
	showtabs=false,                  
	tabsize=2
}
\newenvironment{condicoes}[1][Onde:]
{#1 \begin{tabular}[t]{>{$}l<{$} @{${} \implies {}$} l}}
	{\end{tabular}\\[\belowdisplayskip]}
% Cortesia de Dave: https://tex.stackexchange.com/questions/95838/how-to-write-a-perfect-equation-parameters-description

\lstset{style=mystyle}

% --- Cabeçalho ---

%\author{David Navarro}
\title{Notas de aula: Introdução às funções de uma variável, \\ \large com Gabriel Bondioli Piterutti}
\date{\vspace{-5ex}}
\begin{document}
	\nocite{*}
	\maketitle
	\tableofcontents
	\newpage
	
	\section{Encontro 1 - 08 de julho de 2025}
	
	Matemática é linguagem! Isso significa que nos comunicamos com as outras pessoas e expressamos ideias matemáticas por meio de símbolos.
	
	\subsection{Intervalos}
	
	\begin{snugshade}
		\textcolor{blue}{Relembre!}
		
		Denotamos o conjunto dos números reais pela letra estilizada $\mathbb{R}$, e dentro deste conjunto há ainda outros conjuntos.
		
		\begin{center}
			\begin{tikzpicture}[scale=2]
				
				% Retângulo dos reais
				\draw[fill=gray!10] (0,0) rectangle (6,3);
				\node[anchor=north west] at (6.05,3) {$\mathbb{R}$};
				
				% Retângulo dos racionais (lado esquerdo)
				\draw[fill=blue!10] (0.5,0.5) rectangle (3.8,2.5);
				\node at (3.6,2.2) { $\mathbb{Q}$};
				
				% Retângulo dos irracionais (lado direito)
				%\draw[fill=red!10, opacity=0.7] (3.2,0.5) rectangle (5.5,2.5);
				\node at (5,2.2) {$\mathbb{R} \setminus \mathbb{Q}$};
				
				% Retângulo dos inteiros
				\draw[fill=green!10] (0.9,1) rectangle (3.4,2);
				\node at (3.2,1.8) {$\mathbb{Z}$};
				
				% Retângulo dos naturais
				\draw[fill=orange!10] (1.3,1.2) rectangle (3,1.8);
				\node at (2.65,1.6) {$\mathbb{N}$};
				
				% Título
				\node at (3,3.6) {\textbf{Conjuntos Numéricos}};
				
			\end{tikzpicture}
		\end{center}
		
		Assim, podemos representar os diferentes conjuntos de várias formas diferentes:
		
		\begin{itemize}
			\item[] \textbf{Conjunto dos Naturais:} $\mathbb{N} = \{0, 1, 2, 3, 4, 5, \dots\}$
			\item[] \textbf{Conjunto dos Inteiros:} $\mathbb{Z} = \{\dots, -3, -2, -1, 0, 1, 2, 3, \dots\}$
			\item[] \textbf{Conjunto dos Racionais:} $\displaystyle \mathbb{Q} = \left\{ \frac{a}{b} \,\middle|\, a \in \mathbb{Z},\, b \in \mathbb{Z}^* \right\}$. Lê-se ``\textit{o conjunto de todas as frações $a$ sobre $b$, tal que $a$ pertence ao conjunto dos inteiros, e $b$ pertence ao conjunto dos inteiros diferentes de zero.}''
			
			Exemplos: $\mathbb{Q} = \left\{ -1,\, \frac{1}{2},\, 0,\, 3,\, \frac{22}{7},\, \dots \right\}$
			
			\begin{example}
				Considere $\displaystyle \frac{\sqrt{2}}{2}$. Ele é um número racional? Não, pois $\sqrt{2} \notin \mathbb{Z}$ (lê-se \textit{``raiz de dois não pertence ao conjunto dos números inteiros''}).
				
				E um número qualquer $\displaystyle \frac{a}{b}$, com $b = 0$? Também não, pois para que um número seja racional, $b \neq 0$ (lê-se \textit{``$b$ tem que ser diferente de zero''}).
			\end{example}
			
			\item[] \textbf{Conjunto dos Irracionais:} $\mathbb{R} \setminus \mathbb{Q} = \{\pi,\, \sqrt{2},\, e,\, \dots\}$. A contrabarra ($\setminus$) é um operador de diferença entre conjuntos, e lemos \textit{``o conjunto dos números irracionais é o conjunto dos números reais, menos o conjunto dos números racionais''}.
		\end{itemize}
	\end{snugshade}
	
	
	Um intervalo real é um conjunto de elementos pertencentes ao conjunto dos números reais, que possui dois extremos.
	
	Podemos pensar nos extremos como os dois elementos que limitam esse intervalo, ou ainda, como os elementos que indicam o início e o fim do pedaço dos números reais.
	
	\begin{center}
		\begin{tikzpicture}[scale=1.2]
			
			% Eixo x
			\draw[->] (-3.5,0) -- (4.5,0) node[right] {\small Reta dos reais};
			
			% Pontos extremos
			\draw[thick] (-2,0.1) -- (-2,-0.1);
			\node[below] at (-2,-0.1) {\small $a$};
			
			\draw[thick] (3,0.1) -- (3,-0.1);
			\node[below] at (3,-0.1) {\small $b$};
			
			% Intervalo destacado
			\draw[very thick, color=blue] (-2,0) -- (3,0);
			\draw[fill=blue] (-2,0) circle (2pt); % ponto fechado
			\draw[fill=blue] (3,0) circle (2pt);  % ponto fechado
			
			% Anotação superior
			\node[above, text=blue] at (0.5,0.1) {\footnotesize Intervalo $[a, b]$};
			
			% Setas explicativas
			\draw[->, thick, color=gray] (-2.2,1) -- (-2,0.1);
			\node[above left, text=gray] at (-2.2,1) {\tiny Início};
			
			\draw[->, thick, color=gray] (3.2,1) -- (3,0.1);
			\node[above right, text=gray] at (3.2,1) {\tiny Fim};
			
			% Título
			\node at (0,1.8) {\textbf{Extremos de um intervalo como limites do pedaço dos reais $\mathbb{R}$}};
			
		\end{tikzpicture}
	\end{center}
	
	\textbf{Como denotar um intervalo?} Podemos fazê-lo usando os colchetes ( $\left[ \,\, \right]$ ), observando o seguinte:
	
	\begin{itemize}
		\item Os símbolos $[$ e $]$ indicam que o \textbf{extremo está incluído} no intervalo (chamado de \textbf{intervalo fechado}).
		\item Os símbolos $]\,$ e $[$ indicam que o \textbf{extremo está excluído} do intervalo (chamado de \textbf{intervalo aberto}).
		\item Por exemplo: $[a, b[$ representa todos os reais entre $a$ e $b$, \textbf{incluindo o $a$}, mas \textbf{excluindo o $b$}.
	\end{itemize}
	
	\vspace{0.5cm}
	
	\begin{center}
		\begin{tikzpicture}[scale=1.1]
			% Reta dos reais
			\draw[->] (-2,0) -- (7,0);
			\node[right] at (7,0) {\small $\mathbb{R}$};
			
			% Ponto a (fechado)
			\draw[fill=blue] (2,0) circle (2pt);
			\node[below] at (2,-0.1) {\small $a$};
			
			% Ponto b (aberto)
			\draw[draw=blue, fill=white, thick] (5,0) circle (2pt);
			\node[below] at (5,-0.1) {\small $b$};
			
			% Segmento do intervalo
			\draw[very thick, blue] (2,0) -- (5,0);
			\node[above, text=blue] at (3.5,0.1) {\footnotesize Intervalo $[a, b[$};
			
		\end{tikzpicture}
	\end{center}
	
	\begin{example}Diferentes formas de denotar um intervalo.
		
		\begin{center}
			% Intervalo [1,3]
			\begin{tikzpicture}[scale=1]
				\draw[->] (0,0) -- (7,0);
				\node[right] at (7,0) {\small $\mathbb{R}$};
				\draw[fill=blue] (2,0) circle (2pt);
				\node[below] at (2,-0.1) {\small 1};
				\draw[fill=blue] (5,0) circle (2pt);
				\node[below] at (5,-0.1) {\small 3};
				\draw[very thick, blue] (2,0) -- (5,0);
				\node[above, text=blue] at (3.5,0.1) {\footnotesize $[1,3]$};
			\end{tikzpicture}
			
			\vspace{0.5cm}
			
			% Intervalo ]1,3] = (1,3]
			\begin{tikzpicture}[scale=1]
				\draw[->] (0,0) -- (7,0);
				\node[right] at (7,0) {\small $\mathbb{R}$};
				\draw[draw=blue, fill=white, thick] (2,0) circle (2pt);
				\node[below] at (2,-0.1) {\small 1};
				\draw[fill=blue] (5,0) circle (2pt);
				\node[below] at (5,-0.1) {\small 3};
				\draw[very thick, blue] (2,0) -- (5,0);
				\node[above, text=blue] at (3.5,0.1) {\footnotesize $]1,3]$};
			\end{tikzpicture}
			
			\vspace{0.5cm}
			
			% Intervalo [1,3[ = [1,3)
			\begin{tikzpicture}[scale=1]
				\draw[->] (0,0) -- (7,0);
				\node[right] at (7,0) {\small $\mathbb{R}$};
				\draw[fill=blue] (2,0) circle (2pt);
				\node[below] at (2,-0.1) {\small 1};
				\draw[draw=blue, fill=white, thick] (5,0) circle (2pt);
				\node[below] at (5,-0.1) {\small 3};
				\draw[very thick, blue] (2,0) -- (5,0);
				\node[above, text=blue] at (3.5,0.1) {\footnotesize $[1,3[$};
			\end{tikzpicture}
			
			\vspace{0.5cm}
			
			% Intervalo [1,+∞[
			\begin{tikzpicture}[scale=1]
				\draw[->] (0,0) -- (7,0);
				\node[right] at (7,0) {\small $\mathbb{R}$};
				\draw[fill=blue] (2,0) circle (2pt);
				\node[below] at (2,-0.1) {\small 1};
				\draw[very thick, blue] (2,0) -- (6.5,0);
				\node[above] at (6.5,0.1) {\footnotesize $[1,+\infty[$};
			\end{tikzpicture}
			
			\vspace{0.5cm}
			
			% Intervalo ]−∞,3[ = (-∞,3)
			\begin{tikzpicture}[scale=1]
				\draw[->] (0,0) -- (7,0);
				\node[right] at (7,0) {\small $\mathbb{R}$};
				\draw[draw=blue, fill=white, thick] (5,0) circle (2pt);
				\node[below] at (5,-0.1) {\small 3};
				\draw[very thick, blue] (0.5,0) -- (5,0);
				\node[above] at (1.5,0.1) {\footnotesize $]- \infty,3[$};
			\end{tikzpicture}
		\end{center}
	\end{example}
	
	\subsection{Operações com frações}
	\begin{snugshade}
		\textcolor{blue}{Relembre!}
		
		\textit{\textbf{``Dividir por fração é multiplicar pelo inverso''. Mas, por que?}}
		
		Considere dois números racionais \( \frac{a}{b} \) e \( \frac{c}{d} \), com \( b \neq 0 \) e \( d \neq 0 \).  
		Vamos calcular a divisão entre estes dois números:
		
		
		
		\[
		\frac{\frac{a}{b}}{\frac{c}{d}}
		\]
		
		
		
		Agora vamos multiplicar por 1:
		
		
		
		\[
		\frac{\frac{a}{b}}{\frac{c}{d}} \times \textcolor{red}{1}
		\]
		
		
		
		Mas podemos multiplicar por 1 de uma forma mais esperta. Note que:
		
		
	\begin{align*}
		\textcolor{red}{1 = \frac{\frac{d}{c}}{\frac{d}{c}}}, \quad \text{com } c \neq 0
	\end{align*}
		
		
		
		Isto é verdade porque estamos efetuando a divisão de um número por ele mesmo e, além disso, observe que \( \frac{d}{c} \) é o inverso de \( \frac{c}{d} \). Voltando ao problema inicial:
		
		
		
		\begin{align*}
			\frac{\frac{a}{b}}{\frac{c}{d}} &= \frac{\frac{a}{b}}{\frac{c}{d}} \times \textcolor{red}{1}\\
			&= \frac{\frac{a}{b}}{\frac{c}{d}} \times \textcolor{red}{\frac{\frac{d}{c}}{\frac{d}{c}}} \\
			&= \frac{\frac{a\cdot d}{b \cdot c}}{\frac{\cancel{c} \cdot \cancel{d}}{\cancel{d} \cdot \cancel{c}}}\\
			&= \frac{\frac{a\cdot d}{b \cdot c}}{1}\\
			&= \frac{a\cdot d}{b \cdot c}\\
			&= \frac{ad}{bc}
		\end{align*}
		
		\textbf{Portanto:}
		
		
		
		\begin{align*}
			\frac{\frac{a}{b}}{\frac{c}{d}} = \frac{ad}{bc}
		\end{align*}
		
	\end{snugshade}
	
	\begin{example}Operações com frações
		\begin{itemize}
			\item[a)] $\displaystyle\frac{1}{2} + \frac{3}{5} =$
			
			\begin{align*}
				\frac{1}{2} + \frac{3}{5} &= {\textcolor{gray}{\frac{5}{5}}}\cdot\frac{1}{2} + \frac{3}{5} \cdot {\textcolor{gray}{\frac{2}{2}}} \quad \text{(uma multiplicação esperta: $\frac{5}{5} = \frac{2}{2} = 1$)}\\
				&= \frac{5}{10} + \frac{6}{10} \quad \text{(os denominadores foram igualados)}\\
				&= \frac{11}{10}
			\end{align*}
			
			\item[b)] $\displaystyle\frac{7}{5} \div \frac{3}{4} =$
			
			\begin{align*}
				\frac{7}{5} \div \frac{3}{4} &= \frac{\frac{7}{5}}{\frac{3}{4}} \quad \text{(Dividir por fração é multiplicar pelo inverso)}\\
				&= \frac{7}{5} \cdot \frac{4}{3}\\
				&= \frac{28}{15}
			\end{align*}
		\end{itemize}
	\end{example}
	
	\subsection{Funções de uma variável real}
	
	\textbf{Reconhecendo funções a partir de exemplos.}
	
	Uma função é um tipo especial de relação entre conjuntos em que, para cada elemento de um conjunto, existe apenas um elemento associado pertencente a outro conjunto. Chamamos esse critério de \textit{unicidade}.
	
	Considere dois conjuntos $A$ e $B$. Considere, também, que $A$ possui um elemento $a$, e $B$ possui os elementos $b_1$ e $b_2$. Expressamos essa ideia matemática como $a \in A$ (lê-se \textit{``$a$ pertence ao conjunto $A$''}) e $b_1, b_2 \in B$ (lê-se \textit{``$b_1$ e $b_2$ pertencem ao conjunto $B$''}). 
	
	Veja que a figura abaixo \textbf{não representa uma função}, pois não atende ao critério de \textit{unicidade}.
	
	\begin{center}
		\begin{tikzpicture}[scale=1.3]
			
			% Conjunto amebóide à esquerda (domínio) - Conjunto A
			\draw[fill=blue!10, thick]
			(0,0) .. controls (0.3,1) and (1.8,1.2) .. (2.2,0.5)
			.. controls (2.6,-0.5) and (1.5,-1.3) .. (0.5,-1.2)
			.. controls (-0.3,-0.8) and (-0.3,0.3) .. (0,0);
			\node at (1,-1.8) {\small Conjunto $A$};
			
			% Elemento do conjunto A: a
			\node[fill=blue!60, circle, inner sep=2pt] (a) at (1.2,-0.2) {};
			\node[above left] at (1.2,-0.2) {\small $a$};
			
			% Conjunto amebóide à direita (contradomínio) - Conjunto B
			\draw[fill=orange!10, thick]
			(5,0) .. controls (5.3,1) and (6.8,1.2) .. (7.2,0.5)
			.. controls (7.6,-0.5) and (6.5,-1.3) .. (5.5,-1.2)
			.. controls (4.7,-0.8) and (4.7,0.3) .. (5,0);
			\node at (6.7,-1.8) {\small Conjunto $B$};
			
			% Elementos do conjunto B: b_1 e b_2
			\node[fill=orange!60, circle, inner sep=2pt] (b1) at (6.2,-0.3) {};
			\node[above right] at (6.2,-0.3) {\small $b_1$};
			
			\node[fill=orange!60, circle, inner sep=2pt] (b2) at (6.2, 0.3) {};
			\node[above right] at (6.2, 0.3) {\small $b_2$};
			
			% Seta representando a função
			\draw[->, thick] (a) -- (b1);
			\draw[->, thick] (a) -- (b2);
			
			% Legenda explicativa
			\node at (3.6,-2.2) {\footnotesize O elemento $a$ de $A$ se associa aos elementos $b_1, b_2$ de $B$. \textbf{O diagrama \textcolor{red}{não} representa uma função.}};
			
		\end{tikzpicture}
	\end{center}
	
	Por outro lado, veja que a figura abaixo representa uma função, pois $a \in A$ está relacionado apenas ao elemento $b_1 \in B$.
	
	\begin{center}
		\begin{tikzpicture}[scale=1.3]
			
			% Conjunto amebóide à esquerda (domínio) - Conjunto A
			\draw[fill=blue!10, thick]
			(0,0) .. controls (0.3,1) and (1.8,1.2) .. (2.2,0.5)
			.. controls (2.6,-0.5) and (1.5,-1.3) .. (0.5,-1.2)
			.. controls (-0.3,-0.8) and (-0.3,0.3) .. (0,0);
			\node at (1,-1.8) {\small Conjunto $A$};
			
			% Elemento do conjunto A: a
			\node[fill=blue!60, circle, inner sep=2pt] (a) at (1.2,-0.2) {};
			\node[above left] at (1.2,-0.2) {\small $a$};
			
			% Conjunto amebóide à direita (contradomínio) - Conjunto B
			\draw[fill=orange!10, thick]
			(5,0) .. controls (5.3,1) and (6.8,1.2) .. (7.5,0.5)
			.. controls (7.9,-0.5) and (7.5,-1.2) .. (5.5,-1.1)
			.. controls (4.7,-0.8) and (4.7,0.3) .. (5,0);
			\node at (6.7,-1.8) {\small Conjunto $B$};
			
			% Elemento do conjunto B: b
			% Elementos do conjunto B: b_1 e b_2
			\node[fill=orange!60, circle, inner sep=2pt] (b1) at (6.2,-0.3) {};
			\node[above right] at (6.2,-0.3) {\small $b_1 = f(a)$};
			
			\node[fill=orange!60, circle, inner sep=2pt] (b2) at (6.2, 0.3) {};
			\node[above right] at (6.2, 0.3) {\small $b_2$};
			
			% Seta representando a função
			\draw[->, thick] (a) -- (b1);
			\node[above] at (3.7, -0.2) {\small $f$};
			
			% Legenda explicativa
			\node at (3.6,-2.2) {\footnotesize O elemento $a$ de $A$ se associa ao elemento $b$ de $B$. \textbf{O diagrama representa uma função.}};
			
		\end{tikzpicture}
	\end{center}
	
	Podemos representar a função acima da seguinte forma:
	\begin{align*}
		f: &A \longrightarrow B \quad \text{(lê-se \textit{f de A em B})}\\
		&a \mapsto f(a) = b_1
	\end{align*}
	
	\begin{condicoes}
		f & nome que demos para a função\\
		A, B & os conjuntos\\
		a, b_1 & os elementos dos conjuntos\\
		\mapsto & é um símbolo que \textit{mapeia} os elementos $a$ e $b$
	\end{condicoes}
	
	A regra que relaciona $a \in A$ e $b_1 \in B$ é denotada por $a \mapsto f(a) = b_1$, e dizemos que $f(a)$ é a \textit{imagem} do elemento $a$ pela função $f$.

	
	\textbf{Reconhecendo o domínio, contradomínio e imagem de funções.}
	
	Se os elementos de $A$ e de $B$ são números reais, então dizemos que essa função é de uma variável real a valores reais. Podemos expressar essa ideia matemática da seguinte forma:
	
	\begin{center}
		$A \subseteq \mathbb{R} \quad \text{e} \quad B \subseteq \mathbb{R}$
		
		(Lê-se ``A e B são subconjuntos do conjunto dos reais'')
	\end{center}

	
	Dizemos, ainda, que \textcolor{blue}{$A$ é o domínio da função $f$}, e \textcolor{orange}{$B$ é o contradomínio da função $f$}. Nos referimos ao conjunto de elementos no conjunto $B$ que estão relacionados aos elementos que pertencem a $A$ como imagem da função $f$.
	
	Denotamos o domínio da função $f$ por $D_f$, ou $Dom\,f$, e a imagem por $Im_f$, ou $Im\,f$. No nosso exemplo, em particular, identificamos os conjuntos:
	
	\begin{align*}
		D_f &= A\\
		Im_f &= \{b_1\}\\
		B &\text{ é o contradomínio}
	\end{align*}
	
	\begin{snugshade}
		\textbf{\textcolor{red}{Importante!}}
		
		Existe mais de uma forma de denotar o domínio ($D_f$) e a imagem ($Im_f$) de uma função. Neste material, apresentamos as mais usuais.
		
		A partir do exemplo da função $f$ acima, perceba que $Im_f \subseteq B$ (``o conjunto imagem da função $f$ é um subconjunto de $B$''). Assim, uma outra forma de expressar a imagem de uma função é
		
		\begin{center}
			$Im_f = \{b \in B \,|\, b = f(a), a \in A\}$
			
			(Lê-se ``um elemento $b$ pertence ao conjunto $B$, tal que $b$ é a função $f$ calculada em um $a$, e $a$ pertence ao conjunto $A$'')
		\end{center}
		
		Com essa ideia, podemos denotar funções com elementos infinitos, como aquelas definidas de reais em reais, como a abaixo:
		
		\begin{align*}
			f:\, &\mathbb{R} \longrightarrow \mathbb{R}\\
			&x \mapsto f(x) = y
		\end{align*}
		
		
		\begin{condicoes}
			\mathbb{R} & é o conjunto dos números reais\\
			x & uma variável independente qualquer, pertencente aos reais\\
			f(x) = y & uma variável dependente de $x$, pertencente aos reais\\
			x \mapsto f(x) = y & a regra que relaciona $x$ e $f(x) = y$
		\end{condicoes}
		
		O que essa função $f$ faz é relacionar um $x \in \mathbb{R}$ e um $y \in \mathbb{R}$. Assim,
		\begin{align*}
			D_f &= \mathbb{R}\\
			Im_f &= \{y \in \mathbb{R} \,|\, y = f(x), x \in \mathbb{R}\}\\
			\text{ou ainda, } Im_f &= \{f(x) \,|\, x \in D_f\}
		\end{align*}
		
	\end{snugshade}
	
	\begin{example} Sejam os conjuntos finitos $A = \{1,2,3,4\}$ e $B = \{10, 11, 12, 13, 14, 15\}$, e a função $f: A \longrightarrow B$ representada pelo diagrama abaixo:
		
		\begin{center}
			\begin{tikzpicture}[>=stealth, node distance=1cm]
				
				% Caixas dos conjuntos
				\draw[fill=gray!20, thick] (-2.5,-1) rectangle (-0.5,4); % Caixa A
				\draw[fill=gray!20, thick] (2.5,-2) rectangle (4.5,5);   % Caixa B
				\draw[fill=gray!40, thick] (2.7,-1.5) rectangle (4.3,3); % Caixa Im_f
				
				% Rótulos das caixas
				\node at (-1.5,4.3) {\textbf{Conjunto $A = D_f$}};
				\node at (3.5,5.3) {\textbf{Conjunto $B$}};
				\node at (3.5,2.6) {\small \textbf{$Im_f$}};
				
				% Elementos de A
				\node[fill=black!20, circle, inner sep=2pt] (A1) at (-1.5,3.5) {1};
				\node[fill=black!20, circle, inner sep=2pt] (A2) at (-1.5,2.5) {2};
				\node[fill=black!20, circle, inner sep=2pt] (A3) at (-1.5,1.5) {3};
				\node[fill=black!20, circle, inner sep=2pt] (A4) at (-1.5,0.5) {4};
				
				% Elementos de B
				\node[fill=green!60, circle, inner sep=2pt] (B1) at (3.5,2.0) {10};
				\node[fill=orange!60, circle, inner sep=2pt] (B2) at (3.5,4.5) {11};
				\node[fill=green!60, circle, inner sep=2pt] (B3) at (3.5,1.0) {12};
				\node[fill=orange!60, circle, inner sep=2pt] (B4) at (3.5,3.5) {13};
				\node[fill=green!60, circle, inner sep=2pt] (B5) at (3.5,0) {14};
				\node[fill=green!60, circle, inner sep=2pt] (B6) at (3.5,-1) {15};
				
				% Setas da função f
				\draw[->, thick] (A1) -- (B1); % 1 → 10
				\draw[->, thick] (A2) -- (B3); % 2 → 12
				\draw[->, thick] (A3) -- (B5); % 3 → 14
				\draw[->, thick] (A4) -- (B6); % 4 → 15
				
				% Rótulo da função f
				\node at (0.9,3.1) {\small $f$};
				
			\end{tikzpicture}
		\end{center}
		
		Outra maneira de representar o diagrama da ilustração acima é por meio de uma tabela que relaciona os valores $x$ com seus valores $f(x)$ correspondentes:
		
		\begin{table}[h]
		\caption{Tabela de valores: $x \mapsto f(x)$}
		\centering
		\begin{tabular}[t]{|c|c|}
			\hline
			$x$ & $f(x)$\\
			\hline
			1 & 10\\
			\hline
			2 & 12\\
			\hline
			3 & 14\\
			\hline
			4 & 15\\
			\hline
		\end{tabular}
		\end{table}
		
		Temos:
		
		\begin{align*}
			D_f &= A = \{1,2,3,4\}\\
			Im_f &= \{10, 12, 14, 15\} = \{f(x) \,|\, x \in D_f\} \\
			B &\text{ é o contradomínio}			
		\end{align*}
		
		Observe que os valores $11, 13 \in B$ não estão associados a nenhum $x \in D_f$. Compare os conjuntos $B = \{10, \textcolor{red}{11}, 12, \textcolor{red}{13}, 14, 15\}$ e $Im_f = \{10, 12, 14, 15\}$. Perceba que os valores destacados em vermelho no contradomínio não pertencem à imagem.
		
		Isso significa que o conjunto imagem é um subconjunto do contradomínio. Em notação matemática, $Im_f \subseteq B$. 
	\end{example}
	
	\textbf{Uma função especial: a função constante.}
	
	Agora, imagine dois conjuntos $C$ e $D$, com $c_1, c_2 \in C$ e $d \in D$, de modo que $f: C \longrightarrow D$ e $f(c_1) = d = f(c_2)$ (isto é, todos os elementos de $C$ correspondem a $d \in D$). Temos, assim, uma função constante.
	
	Uma função constante é aquela em que todos os elementos de um conjunto estão relacionados a um mesmo elemento do outro conjunto. Observe:
	
	\begin{center}
		\begin{tikzpicture}[scale=1.3]
			
			% Conjunto amebóide à esquerda (domínio) - Conjunto A
			\draw[fill=blue!10, thick]
			(0,0) .. controls (0.3,1) and (1.8,1.2) .. (2.2,0.5)
			.. controls (2.6,-0.5) and (1.5,-1.3) .. (0.5,-1.2)
			.. controls (-0.3,-0.8) and (-0.3,0.3) .. (0,0);
			\node at (1,-1.8) {\small Conjunto $C$};
			
			% Elementos do conjunto A: a e c
			\node[fill=blue!60, circle, inner sep=2pt] (a) at (1.2,0.3) {};
			\node[above left] at (1.2,0.3) {\small $c_1$};
			
			\node[fill=blue!60, circle, inner sep=2pt] (c) at (1.4,-0.6) {};
			\node[below left] at (1.4,-0.6) {\small $c_2$};
			
			% Conjunto amebóide à direita (contradomínio) - Conjunto B
			\draw[fill=orange!10, thick]
			(5,0) .. controls (5.3,1) and (6.8,1.2) .. (7.2,0.5)
			.. controls (7.6,-0.5) and (6.5,-1.3) .. (5.5,-1.2)
			.. controls (4.7,-0.8) and (4.7,0.3) .. (5,0);
			\node at (6.7,-1.8) {\small Conjunto $D$};
			
			% Elemento do conjunto B: b
			\node[fill=orange!60, circle, inner sep=2pt] (b) at (6.2,-0.2) {};
			\node[above right] at (6.2,-0.2) {\small $d$};
			
			% Setas representando a função constante: a → b, c → b
			\draw[->, thick] (a) -- (b);
			\draw[->, thick] (c) -- (b);
			\node at (3.2, 0.3) {$f$};
			
			% Legenda explicativa
			\node at (3.6,-2.2) {\footnotesize Os elementos $c_1$ e $c_2$ de $C$ se associam ao elemento $d$ de $D$};
			
		\end{tikzpicture}
	\end{center}
	
	\textbf{O plano cartesiano}
	
	Uma forma comum de representar as relações entre dois conjuntos é em um plano cartesiano.
	
	Um plano cartesiano tem dois eixos  perpendiculares ($x$ e $y$) e permite localizarmos pontos de interesse no gráfico, associando valores de $x$ e $y$ na forma de \textit{pares ordenados}.
	
	Um par ordenado tem a forma $(x,y)$, de modo que o par ordenado $(1,3)$ representa o ponto no plano em que $x=1$ e $y=3$.
	
	Alguns autores costumam nomear o eixo vertical do plano cartesiano por $f(x)$, ao invés de $y$. Assim, o par ordenado do exemplo acima também pode ser representado como $(x, f(x)) = (1,3)$.
	
	A ordem em que os elementos aparecem em um par ordenado importa. Observe o exemplo abaixo.
	
	\begin{example}
		Seja o conjunto $A = [1,3]$ (ou seja, um intervalo de valores reais).
		
		Seja a função $f: [1,3] \in A \longrightarrow \{4\}$, tal que $x \mapsto f(x) = y = 4$. Como o conjunto $\{4\}$ tem apenas um elemento, então todo elemento $x \in [1,3]$ levará ao elemento $y=4$.
		
		\begin{figure}[h]
			\centering
			\begin{tikzpicture}
				\begin{axis}[
					axis lines=middle,
					xmin=-4, xmax=4.5,
					ymin=-4, ymax=6,
					xlabel={$x$},
					ylabel={$f(x) = y$},
					xlabel style = {right},
					ylabel style = {above},
					grid=both,
					samples=100,
					clip=true,
					legend style={font=\small}
					]
					
					% Função constante: f(x) = 4 para x no intervalo [1,3]
					\addplot[domain=1:3, thick, blue] {4};
					\addlegendentry{$f(x) = 4$}
					
					% Bolinhas nos extremos do intervalo
					\draw[fill=blue] (axis cs:1,4) circle (2pt);
					\draw[fill=blue] (axis cs:3,4) circle (2pt);
					
				\end{axis}
			\end{tikzpicture}
			\caption{Gráfico da função constante $f(x) = 4$, onde todos os valores de $x$ no intervalo $[1,3]$ são levados ao único valor $y = 4$.}
			\label{fig:funcao_constante_no_plano}
		\end{figure}
		
		O conjunto de pontos do gráfico da Fig.~\ref{fig:funcao_constante_no_plano} pode ser denotado $G_f = \{ (x, f(x)) \,|\, x \in A\}$, e cada elemento deste conjunto é chamado \textit{par ordenado}.
	\end{example}
	
	\begin{snugshade}
		\textbf{\textcolor{violet}{Saiba mais!}}
		
		Podemos pensar no plano cartesiano como um mapa que guarda e representa endereços de vários pontos. Os pares ordenados, por sua vez, são os endereços de cada um dos pontos.
	\end{snugshade}
	
	
	\subsection{Resolvendo exercícios}
	Os exercícios a seguir foram extraídos de Guidorizzi (2012).
	
	\begin{practice}
		
		\textbf{Exercício 2 do livro.} Simplifique $\dfrac{f(x) - f(p)}{x - p}$, $(x \neq p)$ sendo dados:
		
		\begin{enumerate}
			\item[a)] $f(x) = x^2$ e $p = 1$
			
			\begin{align*}
				f(x) &= x^2 \\
				f(1) &= 1^2 = 1 \\
				\\
				\frac{f(x) - f(1)}{x - 1} &= \frac{x^2 - 1}{x - 1} \\
				&= \frac{(x - 1)(x + 1)}{x - 1} \\
				&= x + 1 \quad \text{(para } x \ne 1\text{)}
			\end{align*}
			
			\item[b)] $f(x) = x^2$ e $p = -1$
			
			\begin{align*}
				f(x) &= x^2 \\
				f(-1) &= (-1)^2 = 1 \\
				\\
				\frac{f(x) - f(-1)}{x - (-1)} &= \frac{x^2 - 1}{x + 1} \\
				&= \frac{(x - 1)(x + 1)}{x + 1} \\
				&= x - 1 \quad \text{(para } x \ne -1\text{)}
			\end{align*}
			
			\item[c)] $f(x) = x^2$ e $p$ qualquer
			
			\begin{align*}
				f(x) &= x^2 \\
				f(p) &= p^2 \\
				\\
				\dfrac{f(x) - f(p)}{x - p} &= \dfrac{x^2 - p^2}{x - p} \quad \text{(Diferença entre dois quadrados)}\\
				&= \dfrac{(x - p)(x + p)}{x - p} \\
				&= x + p \quad \text{(para } x \ne p\text{)}
			\end{align*}
			
		\end{enumerate}
		
		\begin{snugshade}
			\textbf{\textcolor{red}{Importante!}}
			
			Na expressão $\dfrac{f(x) - f(p)}{x - p}$, $f(p)$ representa outra função diferente de $f(x)$?
			
			Não. Uma vez definida a função $f$ ela não é alterada. Por exemplo, uma vez que nos foi dito que a função $f$ é dada pela regra $f(x) = x^2$, toda vez que aparecer "f de alguma coisa", iremos pegar a "alguma coisa" e elevar ao quadrado. 
			
			Deste modo, $f(p) = p^2$ e, caso nos tenha sido fornecido algum valor específico para o $p$, o que se faz é uma troca da letra $p$ pelo valor dado. Se nos foi dado que $p = 3$, então a expressão $f(p) = p^2$ ficará $f(3) = 3^2$.
			
			Outra forma de observar isto é que $f(x)$ é um símbolo que representa o cálculo da função $f$ num determinado valor de $x$ e, portanto, o $f(x)$ está dizendo que a regra da função $f$ está sendo aplicada no valor $x$.
		\end{snugshade}
		
	\end{practice}
	
	\pagebreak
	\section{Encontro 2 - 10 de julho de 2025}
	
	\subsection{Função afim}
	Uma função afim tem a forma abaixo:
	
	\begin{equation}
		f(x) = \textcolor{red}{a}x + \textcolor{blue}{b}, a \neq 0
	\end{equation}
	
	\begin{condicoes}
		f(x)=y & variável dependente de $x$\\
		\textcolor{red}{a} & coeficiente angular da reta\\
		\textcolor{blue}{b} & coeficiente linear\\
		x & variável independente    
	\end{condicoes}
	
	\begin{snugshade}
		{\color{red}Importante!}
		
		Toda função afim é de primeiro grau e, portanto, é uma reta. Com dois pontos no plano cartesiano podemos formar uma reta.
	\end{snugshade}
	
	\begin{example}\label{Ex:funcao_afim_1} Considere a função $f(x) = 2x + 1$, com $a \neq 0$, definida no conjunto dos números reais (que denotamos por $\mathbb{R}$).
		Trata-se de uma função afim com coeficiente angular igual a $2$ (ou seja, $a = 2$).
		
		O gráfico da função $f$ é representado no plano cartesiano da Fig.~\ref{fig:funcao_afim}
		.
		\begin{figure}[h]
			\centering
			\begin{tikzpicture}
				\begin{axis}[xmax=5, 
					xmin=-5, 
					ymax=5, 
					ymin=-5, 
					axis lines=middle, 
					legend pos=south east, 
					xlabel={$x$},
					ylabel={$y$},
					minor tick num=1,
					xlabel style = {right},
					ylabel style = {above},
					grid=both,
					legend style={font=\tiny}]
					\addplot[smooth, color=blue]{2*x+1};
					\addlegendentry{$f(x)=2x+1$};
				\end{axis}
			\end{tikzpicture}
			\caption{Gráfico da função $f(x) = 2x + 1$.}
			\label{fig:funcao_afim}
		\end{figure}
	\end{example}
	
	O que o gráfico da Fig.~\ref{fig:funcao_afim} nos mostra? Vamos ver o todo e depois resolver passo a passo.
	
	\begin{tabular}{cc}
		Quando $x=0$,  & $f(0) = 2 \cdot 0 + 1 = 1$ (colocamos $0$ no lugar do $x$)\\
		Quando $y = 0$, & $x = -\displaystyle\frac{1}{2}$ (resolvemos $f(x) = 2x + 1 = 0$)
	\end{tabular}
	
	Resolvendo passo a passo, temos:
	
	\textbf{Quando $x=0$:}
	\begin{align*}
		f(0) &= 2 \cdot 0 + 1\\
		&= 0 + 1\\
		&= 1
	\end{align*}
	Em outras palavras, quando $x=0$, a reta intercepta o eixo $y$ em $y=1$.
	
	\textbf{Quando $y=0$:}
	\begin{align*}
		f(x) = 2x + 1 &= 0\\
		= 2x &= -1\\
		= x &= -\frac{1}{2}
	\end{align*}
	Isso significa que a reta cruzará o eixo $x$ quando $x= \displaystyle -\frac{1}{2}$.
	
	\textbf{Analisando o coeficiente angular.}\label{tema:coef_angular}
	
	Volte à função $f(x) = \textcolor{red}{2}x + 1$. O que o coeficiente angular ($\textcolor{red}{2}$) faz com a reta no gráfico da Fig.~\ref{fig:funcao_afim}?
	
	Observe no gráfico que, conforme caminhamos em uma unidade para a direita na reta dos valores para $x$, $y$ aumenta em duas unidades.
	
	\begin{example}\label{ex:funcao_afim_2}Comparando o comportamento da função para dois valores de $x$. Considere $x = 0$, então $y = 1$. Para $x=1$, temos $y=3$ (aumentamos $y$ em duas unidades).
	\end{example}
	
	Observe também que nessa (e em qualquer outra!) função afim podemos identificar um triângulo retângulo, como o destacado na Fig.~\ref{fig:funcao_afim_2}.
	
	\begin{figure}[h]
		\centering
		\begin{tikzpicture}
			\begin{axis}[xmax=5, 
				xmin=-5, 
				ymax=5, 
				ymin=-5, 
				axis lines=middle, 
				legend pos=south east, 
				xlabel={$x$},
				ylabel={$y$},
				minor tick num=1,
				xlabel style = {right},
				ylabel style = {above},
				grid=both,
				legend style={font=\tiny}]
				\addplot[smooth, color=blue]{2*x+1};
				\addlegendentry{$f(x)=2x+1$};
				
				\addplot[fill=red, draw=red] coordinates { (-0.5,0) (0,1) (0,0) };
			\end{axis}
		\end{tikzpicture}
		\caption{Efeito do coeficiente angular no gráfico da função $f(x) = 2x + 1$.}
		\label{fig:funcao_afim_2}
	\end{figure}
	
	Vamos ampliar o triângulo e analisá-lo na Fig.~\ref{fig:funcao_afim_3}. Note que a base tem tamanho igual a $\displaystyle \frac{1}{2}$, e formamos um ângulo que chamamos de $\theta$ (lê-se \textit{téta}).
	
	\begin{figure}[h]
		\centering
		\begin{tikzpicture}[scale=3]
			% Vértices
			\coordinate (A) at (0,0);            % Ângulo reto (origem)
			\coordinate (B) at (0.5,0);          % Base
			\coordinate (C) at (0.5,1);          % Altura
			
			% Triângulo vermelho
			\filldraw[fill=red!50, draw=red, thick] (A) -- (B) -- (C) -- cycle;
			
			% Marcação de ângulo reto
			\draw ($(B)+(-0.1,0)$) -- ($(B)+(-0.1,0.1)$) -- ($(B)+(0,0.1)$);
			
			% Rótulos dos pontos
			\node[below left] at (A) {$A$};
			\node[below right] at (B) {$B$};
			\node[above right] at (C) {$C$};
			
			% Indicação do ângulo theta
			\node at ($(A)+(0.12,0.12)$) {$\theta$};
			
			% Rótulo entre A e B (base)
			\path (A) -- (B) node[midway, below] {$\frac{1}{2}$};
			
			% Rótulo entre B e C (altura)
			\path (B) -- (C) node[midway, right] {$1$};
		\end{tikzpicture}
		\caption{Esse é o triângulo destacado da função.}
		\label{fig:funcao_afim_3}
	\end{figure}
	
	Podemos determinar o valor do ângulo $\theta$ usando a função trigonométrica tangent ($\tan$) e, da seguinte forma:
	\begin{align*}
		\tan{\theta} &= \frac{\text{cateto oposto}}{\text{cateto adjacente}}\\
		\tan{\theta} &= \frac{1}{\displaystyle\frac{1}{2}} \text{ \textit{(dividir por fração é o mesmo que multiplicar pelo inverso)}}\\
		\tan{\theta} &= 1 \cdot \frac{2}{1}\\
		\tan{\theta} &= 2\\
	\end{align*}
	
	\begin{snugshade}
		Repare como o coeficiente angular da função $f(x) = \textcolor{red}{2}x + 1$, $\textcolor{red}{a = 2}$, é igual a $\tan{\theta}$. Veja também que ele é maior que zero. Matematicamente, expressamos assim: $\tan{\theta} > 0$.
	\end{snugshade}
	
	O que acontece se mudarmos o sinal da $\tan{\theta}$? A inclinação da reta mudará. Dizemos que, quando \textcolor{blue}{$a > 0$}, a função cresce, e; quando \textcolor{red}{$a < 0$}, a função decresce. Já quando $a=0$, a função será constante, já que a reta não possui inclinação.
	
	\begin{figure}[h]
		\centering
		% Gráfico com a > 0
		\begin{subfigure}[t]{0.45\textwidth}
			\centering
			\begin{tikzpicture}
				\begin{axis}[
					title={$a > 0$},
					axis lines=middle,
					xlabel={$x$},
					ylabel={$y$},
					grid=both,
					xmin=-2, xmax=2,
					ymin=-3, ymax=5,
					xlabel style = {right}
					]
					\addplot[domain=-2:2, blue, thick]{2*x + 1};
				\end{axis}
			\end{tikzpicture}
			\caption{Inclinação positiva: reta crescente}
		\end{subfigure}
		\hspace{1cm}
		% Gráfico com a < 0
		\begin{subfigure}[t]{0.45\textwidth}
			\centering
			\begin{tikzpicture}
				\begin{axis}[
					title={$a < 0$},
					axis lines=middle,
					xlabel={$x$},
					ylabel={$y$},
					grid=both,
					xmin=-2, xmax=2,
					ymin=-3, ymax=5,
					xlabel style = {right}
					]
					\addplot[domain=-2:2, red, thick]{-2*x + 1};
				\end{axis}
			\end{tikzpicture}
			\caption{Inclinação negativa: reta decrescente}
		\end{subfigure}
		\caption{Comparação da inclinação da reta tangente para coeficiente angular positivo e negativo}
	\end{figure}
	
	\pagebreak
	\subsection{Função quadrática}
	As funções quadráticas tem a seguinte forma:
	\begin{equation}
		f(x) = ax^2 + bx + c, a \neq 0
	\end{equation}
	
	\begin{condicoes}
		f(x)=y & variável dependente de $x$\\
		a & termo dominante\\
		c & termo independente\\
		x & variável independente
	\end{condicoes}
	
	Elas são conhecidas pelos seus comportamentos em forma de parábola, com concavidades para cima ou para baixo.
	
	\begin{snugshade}
		\textbf{\textcolor{violet}{Saiba mais!}}
		
		Qual o melhor método para resolver uma equação quadrática? Aquela que funciona e que você não erra!
		
		Outros métodos para resolver equações quadráticas incluem:
		
		\begin{itemize}
			\item Fórmula quadrática (o popular Bhaskara):
			
				\begin{align*}
					\text{Para uma função } & f(x) = ax^2 + bx + c\\
					\Delta &= b^2 - 4ac\\
					&\text{Calculando $x_1$ e $x_2$:}\\
					x_{1,2} &= \frac{- b^2 \pm \sqrt{\Delta}}{2 \cdot a}
				\end{align*}
				
				Também podemos calcular de uma maneira mais direta:
				
				\begin{align*}
					x_{1,2} &= \frac{- b^2 \pm \sqrt{b^2 - 4ac}}{2 \cdot a}
				\end{align*}
			
			\item Soma e produto;
			\item Fatoração;
			\item Completar quadrados.
		\end{itemize}
	\end{snugshade}
	
	\textbf{Analisando o comportamento da função, da mesma forma como fizemos com a função afim.}
	
	\begin{example}\label{ex:f_x2-6x+8}
		Seja a função $f(x) = x^2 -6x + 8$, definida no conjunto $\mathbb{R}$.
		
		Quando $x=0$, temos 
		
		\begin{align*}
			f(x) &= x^2 -6x + 8 \\
			f(0) &=  0^2 -6 \cdot 0 + 8 \\
			&= 0 + 0 + 8 \\
			&= 8
		\end{align*}
		Isso significa que, quando $x=0$, a função intercepta o eixo $y$ em $y=8$.
		
		Por outro lado, quando $f(x) = 0$, teremos
		
		\begin{align*}
			f(x) &= x^2 -6x + 8 \\
			0 &= x^2 -6x + 8 \\
			x &= \frac{-(-6) \pm \sqrt{(-6)^2 - 4 \cdot 1 \cdot 8}}{2 \cdot 1} \quad &\text{(Fórmula quadrática)}\\
			x &= \frac{6 \pm \sqrt{36 - 32}}{2} \\
			x &= \frac{6 \pm \sqrt{4}}{2} \\
			x &= \frac{6 \pm 2}{2}
		\end{align*}
		
		Achando as raízes, vemos que a função tocará o eixo $x$ em dois pontos: quando $x=4$ e $x=2$.
		\begin{align*}
			x_1 &= \frac{6 + 2}{2} = \frac{8}{2} = 4 \\
			x_2 &= \frac{6 - 2}{2} = \frac{4}{2} = 2
		\end{align*}
		O gráfico da função quadrática pode ser visto na Fig.~\ref{fig:pontos-funcao-quadratica}.
		
		\begin{figure}[h]
			\centering
			\begin{tikzpicture}
				\begin{axis}[
					axis lines=middle,
					xlabel={$x$},
					ylabel={$y$},
					xmin=-5, xmax=5,
					ymin=-3, ymax=10,
					grid=both,
					width=10cm, height=7cm,
					legend style={at={(0.4,1)}, anchor=north east, font=\footnotesize}
					]
					% Raízes x_1 = 2, x_2 = 4
					\addplot[only marks, mark=*, blue] coordinates {(2,0) (4,0)};
					\addlegendentry{Raízes $x_1 = 2$, $x_2 = 4$};
					
					% O gráfico da função
					\addplot[smooth]{x^2 - 6*x + 8};
					
					% Ponto C = f(0) = 8
					\addplot[only marks, mark=*, red] coordinates {(0,8)};
				\end{axis}
			\end{tikzpicture}
			\caption{Pontos destacados da função $f(x) = x^2 - 6x + 8$: interseção com eixo $y$ e raízes.}
			\label{fig:pontos-funcao-quadratica}
		\end{figure}
	\end{example}
	
	\textbf{Analisando o discriminante (nós o denotamos por $\Delta$)}.
	
	Espere aí... A parábola sempre cortará o eixo $x$ em qualquer função quadrática? Não! Observe o exemplo a seguir:
	
	\begin{example}
		Seja $g(x) = x^2 - 6x + 9$ uma função definida no conjunto $\mathbb{R}$.
		O gráfico da Fig.~\ref{fig:funcao_quadratica} mostra o comportamento dessa função.
		\begin{figure}[h]
			\centering
			\begin{tikzpicture}
				\begin{axis}[
					xmax=5, xmin=-5,
					ymax=5, ymin=-5,
					axis lines=middle,
					legend pos=south east,
					xlabel={$x$},
					ylabel={$y$},
					minor tick num=1,
					xlabel style={right},
					ylabel style={above},
					grid=both,
					legend style={font=\tiny}
					]
					\addplot[smooth, color=blue]{x^2 - 6*x + 9};
					\addlegendentry{$g(x) = x^2 - 6x + 9$}
					
					\addplot[only marks, color=red, mark=*] coordinates {(3,0)};
				\end{axis}
			\end{tikzpicture}
			\caption{Gráfico da função quadrática $g(x) = x^2 - 6x + 9$.}
			\label{fig:funcao_quadratica}
		\end{figure}
		
		Calculando o discriminante ($\Delta$), temos:
		\begin{align*}
			g(x) &= x^2 - 6x + 9 \\
			a &= 1, \quad b = -6, \quad c = 9 \\
			\Delta &= b^2 - 4ac \\
			&= (-6)^2 - 4 \cdot 1 \cdot 9 \\
			&= 36 - 36 \\
			&= 0
		\end{align*}
		
		Experimente encontrar as raízes da função e verá que $\sqrt{\Delta} = 0$, e assim obteremos apenas um valor para x. Em outras palavras, a função tocará o eixo $x$ em apenas um ponto:
		
		\begin{align*}
			x &= \frac{-b \pm \sqrt{\Delta}}{2a} \\
			x &= \frac{-(-6) \pm \sqrt{0}}{2 \cdot 1} \\
			x &= \frac{6 \pm 0}{2} \\
			x &= \frac{6}{2} \\
			x &= 3
		\end{align*}
		
	\end{example}
	
	Em outros casos, a função pode não tocar o eixo $x$. Isso acontece quando $\Delta < 0$ (lê-se \textit{delta é menor que zero}).
	\begin{example}
		Seja a função $h(x) = x^2 - 6x + 10$. Resolvendo da mesma maneira como as demais, temos
		
		\begin{align*}
			h(x) &= x^2 - 6x + 10 \\
			a &= 1,\quad b = -6,\quad c = 10 \\
			\Delta &= b^2 - 4ac \\
			&= (-6)^2 - 4 \cdot 1 \cdot 10 \\
			&= 36 - 40 \\
			&= -4 \\
			x &= \frac{-b \pm \sqrt{\Delta}}{2a} \\
			x &= \frac{-(-6) \pm \sqrt{-4}}{2 \cdot 1} \\
		\end{align*}
		
		Por enquanto não sabemos resolver $\sqrt{-4}$, mas aqui vai o gráfico que mostra o comportamento dessa função (veja a Fig.~\ref{fig:funcao_h}, na página~\pageref{fig:funcao_h}).
		
		\begin{figure}[h]
			\centering
			\begin{tikzpicture}
				\begin{axis}[
					xmax=5, xmin=-5,
					ymax=5, ymin=-5,
					axis lines=middle,
					legend pos=south east,
					xlabel={$x$},
					ylabel={$y$},
					minor tick num=1,
					xlabel style={right},
					ylabel style={above},
					grid=both,
					legend style={font=\tiny}
					]
					\addplot[smooth, color=blue]{x^2 - 6*x + 10};
					\addlegendentry{$h(x) = x^2 - 6x + 10$}
					\addplot[only marks, color=red, mark=*] coordinates {(3,1)};
					
				\end{axis}
			\end{tikzpicture}
			\caption{Gráfico da função quadrática $h(x) = x^2 - 6x + 10$.}
			\label{fig:funcao_h}
		\end{figure}
	\end{example}
	
	Assim, podemos resumir que:
	
	\begin{itemize}
		\item Quando $\Delta > 0$, a função quadrática terá duas raízes reais e distintas, tocando o eixo $x$ em dois pontos;
		\item Quando $\Delta = 0$, a função tocará o eixo $x$ apenas uma vez;
		\item Quando $\Delta < 0$, a função não terá raízes reais, de modo que não tocará o eixo $x$.
	\end{itemize}
	
	\begin{figure}[h]
		\centering
		\begin{tikzpicture}
			\begin{axis}[
				xmin=-1, xmax=6,
				ymin=-2, ymax=12,
				axis lines=middle,
				legend pos=north east,
				xlabel={$x$},
				ylabel={$y$},
				minor tick num=1,
				xlabel style={right},
				ylabel style={above},
				grid=both,
				legend style={font=\tiny}
				]
				% f(x) = x^2 - 6x + 8 → Δ > 0
				\addplot[smooth, color=blue, thick]{x^2 - 6*x + 8};
				\addlegendentry{$f(x) = x^2 - 6x + 8$ \quad $\Delta = 4$}
				
				% g(x) = x^2 - 6x + 9 → Δ = 0
				\addplot[smooth, color=red, thick]{x^2 - 6*x + 9};
				\addlegendentry{$g(x) = x^2 - 6x + 9$ \quad $\Delta = 0$}
				
				% h(x) = x^2 - 6x + 10 → Δ < 0
				\addplot[smooth, color=orange, thick]{x^2 - 6*x + 10};
				\addlegendentry{$h(x) = x^2 - 6x + 10$ \quad $\Delta = -4$}
			\end{axis}
		\end{tikzpicture}
		\caption{Comparação gráfica das funções quadráticas com diferentes valores de $\Delta$.}
		\label{fig:comparacao-delta}
	\end{figure}
	
	\textbf{Analisando a concavidade ($a > 0$, $a < 0$).}
	
	Como sabíamos que a concavidade da parábola estava voltada para cima? Graças ao coeficiente $a$, que nos exemplos anteriores era positivo ($a > 0$).
	
	Observe no gráfico da Fig.~\ref{fig:concavidade-a} da página~\pageref{fig:concavidade-a} como o sinal do coeficiente $a$ afeta a concavidade da parábola.
	
	\begin{figure}[h]
		\centering
		\begin{minipage}{0.48\textwidth}
			\centering
			\begin{tikzpicture}
				\begin{axis}[
					xmin=-2, xmax=5,
					ymin=-5, ymax=10,
					axis lines=middle,
					grid=both,
					xlabel={$x$},
					ylabel={$y$},
					title={$a > 0$},
					legend pos=south east,
					legend style={font=\tiny}
					]
					\addplot[smooth, color=blue, thick]{x^2 - 4*x + 3};
					\addlegendentry{$f(x) = x^2 - 4x + 3$}
				\end{axis}
			\end{tikzpicture}
		\end{minipage}
		\hfill
		\begin{minipage}{0.48\textwidth}
			\centering
			\begin{tikzpicture}
				\begin{axis}[
					xmin=-2, xmax=5,
					ymin=-10, ymax=5,
					axis lines=middle,
					grid=both,
					xlabel={$x$},
					ylabel={$y$},
					title={$a < 0$},
					legend pos=south east,
					legend style={font=\tiny}
					]
					\addplot[smooth, color=red, thick]{-x^2 + 4*x - 3};
					\addlegendentry{$f(x) = -x^2 + 4x - 3$}
				\end{axis}
			\end{tikzpicture}
		\end{minipage}
		\caption{Comparação da concavidade: à esquerda com $a > 0$ (parábola voltada para cima) e à direita com $a < 0$ (parábola voltada para baixo).}
		\label{fig:concavidade-a}
	\end{figure}
	
	
	\pagebreak
	
	\subsection{Função polinomial}
	Uma função polinomial tem a forma
	
	\begin{equation}
		f(x) = a_nx^n + a_{n-1}x^{n-1} + a_{n-2}x^{n-2} \cdots a_2x^2 + a_1x^1 + a_0, n \in \mathbb{N} 
	\end{equation}
	
	\begin{condicoes}
		a_n & termo dominante (aquele que multiplica a variável independente de maior grau)\\
		a_0 & termo independente\\
		x & variável independente\\
		f(x)=y & variável dependente 
	\end{condicoes}
	
	\begin{snugshade}
		\textcolor{red}{Importante!}
		
		Preste especial atenção à definição: para que uma função seja polinomial, o expoente deve ser um número natural. 
	\end{snugshade}
	
	\subsection{Funções racionais}
	Funções racionais tem a forma
	
	\begin{equation}
		f(x) = \frac{p(x)}{q(x)}, q(x) \neq 0, \text{ $p(x)$ e $q(x)$ são funções polinomiais}
	\end{equation}
	
	\begin{example}
		Seja a função $f(x) = \displaystyle \frac{x^2 - 2x - 1}{2x-1}$. Recorde-se de que $2x-1 \neq 0$. Assim, temos
		
		\begin{align*}
			2x - 1 &\neq 0 \quad \text{(Jamais dividirás por zero!)}\\
			2x &\neq 1\\
			x &\neq \frac{1}{2}
		\end{align*}
	\end{example}
	
	\begin{example} A função a seguir \textbf{não} é uma função racional, já que o numerador não é um polinômio. Observe que o expoente $\displaystyle\frac{1}{2} \notin \mathbb{N}$ (lê-se \textit{meio não pertence ao conjunto dos números naturais}).
		
		\begin{equation*}
			f(x) = \frac{\sqrt{x-1}}{x^2+1} = \frac{(x-1)^{\frac{1}{2}}}{x^2+1}
		\end{equation*}
	\end{example}
	
	\subsection{Função exponencial}
	
	\begin{snugshade}
		\textbf{\textcolor{blue}{Relembre}} alguns conceitos sobre as potências:
		
		Observe os exemplos a seguir:
		
		\begin{itemize}
			\item $2^3 = 2 \cdot 2 \cdot 2 = 8$
			\item $(-2)^3 = (-2) \cdot (-2) \cdot (-2) = -8$ (cuidado com os sinais negativos)
			\item $(-2)^4 = (-2) \cdot (-2) \cdot (-2) \cdot (-2) = 16$ (como o expoente é par, o resultado é positivo)
			\item $4^\frac{1}{2} = \sqrt{4} = 2$ 
			\item $\displaystyle 2^{-1} = \frac{1}{2^1} = \frac{1}{2}$
			\item $\displaystyle 0^{-1} = \frac{1}{0^1} = \frac{1}{0}$ (Opa! Não é possível realizar a divisão por zero.)
		\end{itemize}
		
		
		Qualquer número elevado a uma fração pode ser escrito como uma raiz:
		\begin{equation*}
			b^{\frac{{\textcolor{blue}{n}}}{{\textcolor{red}{d}}}} = \sqrt[{\textcolor{red}{d}}]{b^{\textcolor{blue}{n}}}
		\end{equation*}
		
		Observe o que acontece quando $b<0$.
		\begin{equation*}
			(-b)^{\frac{{\textcolor{blue}{n}}}{{\textcolor{red}{d}}}} = \sqrt[{\textcolor{red}{d}}]{(-b)^{\textcolor{blue}{n}}}
		\end{equation*}
		
		Quando $b<0$, a função não está definida para todo $\frac{n}{d} \in \mathbb{R}$. Em outras palavras, quando $b<0$, o gráfico dessa função é cheio de buracos.
	\end{snugshade}
	
	Uma função exponencial tem a forma
	
	\begin{equation}
		f(x) = b^x, \text{ com } b > 0, b \neq 1 
	\end{equation}
	
	\begin{condicoes}
		b & base\\
		x & variável independente
	\end{condicoes}
	
	Observe como a variável independente ($x$) agora é o expoente nessa função.
	
	\subsection{Função logarítmica}
	
	\begin{snugshade}
		\textbf{\textcolor{blue}{Relembre!}}
		
		\begin{definition}[Logaritmo]
			Sejam dois números $a$ e $b$, ambos reais e positivos, com $b \neq 1$. Dizemos que o logaritmo de $a$ na base $b$ é um número $c$ se, e somente se, $b$ elevado a $c$ for igual a $a$.
			Podemos escrever essa ideia matemática da seguinte forma:
			
			\begin{equation*}
				\log_ba = c \iff b^c = a
			\end{equation*}        
		\end{definition}
		
		Assim, o logaritmo responde a pergunta: ``devo elevar a base $b$ a qual número $c$ para obter $a$?
		
		Por exemplo, $\log_{\textcolor{blue}{2}}\textcolor{purple}{4} = \textcolor{red}{2}
		\quad \iff \quad
		\textcolor{blue}{2}^{\textcolor{red}{2}} = \textcolor{purple}{4}$
	\end{snugshade}
	
	
	Uma função logarítmica tem a forma
	
	\begin{equation}
		f(x) = \log_bx, \text{ com } b > 0, b \neq 1
	\end{equation}
	
	\subsection{Resolvendo exercícios}
	Os exercícios a seguir foram extraídos de Guidorizzi (2014).
	
	\begin{practice}
		(Exercício 4 do livro) Dê o domínio da função e esboce o gráfico.
		
		\begin{enumerate}
			\item[a)] $f(x) = 3x$
			
			\begin{align*}
				f(x) &= 3x \\
				\text{Domínio: } &\mathbb{R} \quad \text{(afim, definida para todos os reais)} \\
				\text{Imagem: } &\mathbb{R} \quad \text{(sem restrições)}
			\end{align*}
			
			\begin{figure}[h]
				\centering
				\begin{tikzpicture}
					\begin{axis}[axis lines=middle, grid=both]
						\addplot[blue, thick]{3*x};
						\addlegendentry{$f(x) = 3x$}
					\end{axis}
				\end{tikzpicture}
				\caption{Função afim $f(x) = 3x$}
			\end{figure}
			
			\item[b)] $g(x) = -x + 1$
			
			\begin{align*}
				g(x) &= -x + 1 \\
				a &= -1,\quad b = 1 \\
				\text{Domínio: } & \mathbb{R} \\
				\text{Imagem: } & \mathbb{R}
			\end{align*}
			
			\begin{figure}[h]
				\centering
				\begin{tikzpicture}
					\begin{axis}[
						xmin=-5, xmax=5,
						ymin=-5, ymax=5,
						axis lines=middle,
						legend pos=south east,
						grid=both,
						xlabel={$x$},
						ylabel={$y$},
						minor tick num=1,
						xlabel style={right},
						ylabel style={above},
						legend style={font=\tiny}
						]
						\addplot[smooth, color=blue]{-x + 1};
						\addlegendentry{$g(x) = -x + 1$}
					\end{axis}
				\end{tikzpicture}
				\caption{Função afim $g(x) = -x + 1$.}
				\label{fig:gx_negativo}
			\end{figure}
			
			\item[d)] $f(x) = 2x + 1$
			
			\begin{align*}
				f(x) &= 2x + 1 \\
				a &= 2,\quad b = 1 \\
				\text{Domínio: } & \mathbb{R} \\
				\text{Imagem: } & \mathbb{R}
			\end{align*}
			
			\begin{figure}[h]
				\centering
				\begin{tikzpicture}
					\begin{axis}[
						xmin=-5, xmax=5,
						ymin=-5, ymax=5,
						axis lines=middle,
						legend pos=south east,
						grid=both,
						xlabel={$x$},
						ylabel={$y$},
						minor tick num=1,
						xlabel style={right},
						ylabel style={above},
						legend style={font=\tiny}
						]
						\addplot[smooth, color=blue]{2*x + 1};
						\addlegendentry{$f(x) = 2x + 1$}
					\end{axis}
				\end{tikzpicture}
				\caption{Função afim $f(x) = 2x + 1$.}
				\label{fig:fx_afim}
			\end{figure}
			
			\item[j)] $g(x) = 
			\begin{cases}
				x       & \text{se } x \leq 2 \\
				3       & \text{se } x > 2
			\end{cases}$
			
			\begin{align*}
				g(x) = 
				\begin{cases}
					x & \text{se } x \leq 2 \\
					3 & \text{se } x > 2
				\end{cases}
			\end{align*}
			
			\begin{figure}[h]
				\centering
				\begin{tikzpicture}
					\begin{axis}[
						xmin=-2, xmax=5,
						ymin=0, ymax=5,
						axis lines=middle,
						legend pos=south east,
						grid=both,
						xlabel={$x$},
						ylabel={$y$},
						minor tick num=1,
						xlabel style={right},
						ylabel style={above},
						legend style={font=\tiny}
						]
						% Parte 1: x <= 2
						\addplot[domain=-2:2,smooth, color=blue]{x};
						
						% Parte 2: x > 2
						\addplot[domain=2.01:5,smooth, color=blue]{3};
						
						% Ponto fechado em (2,2)
						\addplot[only marks, mark=*, mark options={blue}, color=blue] coordinates {(2,2)};
						
						% Ponto aberto em (2,3)
						\addplot[only marks, mark=o, mark options={blue}, color=blue] coordinates {(2,3)};
						
						\addlegendentry{$g(x)$}
					\end{axis}
				\end{tikzpicture}
				\caption{Função definida por partes $g(x)$.}
				\label{fig:gx_partes_pontos}
			\end{figure}
			
			
			\item[o)] $f(x) = \displaystyle \frac{x^2 - 1}{x - 1}$
			
			\begin{align*}
				f(x) &= \frac{x^2 - 1}{x - 1} \quad \text{(Repare a diferença de quadrados no numerador)}\\
				&= \frac{\cancel{(x - 1)}(x + 1)}{\cancel{(x - 1)}} \\
				&= x + 1,\quad \text{com } x \neq 1 \\
				\text{Domínio: } & \mathbb{R} \setminus \{1\} \\
			\end{align*}
			
			\begin{figure}[h]
				\centering
				\begin{tikzpicture}
					\begin{axis}[
						xmin=-2, xmax=4,
						ymin=-2, ymax=6,
						axis lines=middle,
						legend pos=south east,
						grid=both,
						xlabel={$x$},
						ylabel={$y$},
						minor tick num=1,
						xlabel style={right},
						ylabel style={above},
						legend style={font=\tiny}
						]
						\addplot[smooth, color=blue, domain=-2:0.99]{x + 1};
						\addplot[smooth, color=blue, domain=1.01:4]{x + 1};
						\addlegendentry{$f(x) = \frac{x^2 - 1}{x - 1}$}
						\addplot[only marks, mark=*, color=red] coordinates {(1,2)};
					\end{axis}
				\end{tikzpicture}
				\caption{Função racional com um ``buraco'' em $x=1$.}
				\label{fig:fx_racional}
			\end{figure}
		\end{enumerate}
	\end{practice}
	
	\begin{practice}
		(Exercício 9 do livro) Determine o domínio.
		
		\item[e)] $h(x) = \sqrt{x + 2}$
		
		\begin{align*}
			h(x) &= \sqrt{x + 2} \\
			\text{Domínio: } & x + 2 \geq 0 \\
			& x \geq -2 \\
			\text{Logo, } \text{Domínio: } & [-2, \infty[
		\end{align*}
		
		O intervalo pode ser representado também em uma reta orientada, em que $x \in \mathbb{R}$.
		
		\begin{center}
			\begin{tikzpicture}
				\begin{axis}[
					axis x line=middle,
					axis y line=none,
					xmin=-4, xmax=6,
					ymin=-1, ymax=1,
					xtick={-2},
					xticklabels={$x=-2$},
					xlabel={$x$},
					grid=none,
					ticks=none,
					xlabel style={right},
					width=10cm, height=4cm
					]
					
					% Intervalo pintado: x >= -2
					\addplot[domain=-2:6, thick, color=blue]{0};
					
					% Ponto fechado em x = -2
					\addplot[only marks, mark=*, color=blue] coordinates {(-2,0)};
					\node[below] at (axis cs:-2,0) {$-2$};
					
				\end{axis}
			\end{tikzpicture}
		\end{center}
		\item[g)] $y = \sqrt{\dfrac{x - 1}{x + 1}}$
		
		\begin{align*}
			y &= \sqrt{\frac{x - 1}{x + 1}} \\
			\text{Para que } y \in \mathbb{R},\quad &\frac{x - 1}{x + 1} \geq 0 \\
			\text{Iremos estudar o sinal da expressão:} \quad &\frac{x - 1}{x + 1} \geq 0 \\
			\text{Restrição no numerador: } &x - 1 \geq 0 \Rightarrow x \geq 1 \\
			\text{Restrição no denominador: } &x + 1 > 0 \Rightarrow x > -1\\
		\end{align*}
		
		Utilizaremos um recurso gráfico para analisar os sinais, popularmente conhecido como ``varal de sinais'' ou ``varalzinho''. 
		
		Nas duas primeiras linhas, analisamos cada parte da expressão, individualmente, nos pontos onde encontramos as restrições ($-1$ e $1$). Na última linha, analisamos o sinal da expressão toda. Lembre-se: queremos o(s) intervalo(s) onde $x \geq 0$, ou seja, onde a reta não for negativa.
		
		\begin{center}
			\begin{tikzpicture}[scale=1.1]
				
				% Linha 1: x - 1
				\node at (-6,3.9) {\small $x - 1$};
				\draw[->] (-5,3.9) -- (5.5,3.9);
				\foreach \x/\lab/\cor in {-3/-/red, 0/-/red, 2/+/blue, 4/+/blue}
				\node[below, text=\cor] at (\x,3.9) {\lab};
				
				% Linha 2: x + 1
				\node at (-6,2.5) {\small $x + 1$};
				\draw[->] (-5,2.5) -- (5.5,2.5);
				\foreach \x/\lab/\cor in {-3/-/red, 0/+/blue, 2/+/blue, 4/+/blue}
				\node[below, text=\cor] at (\x,2.5) {\lab};
				
				% Linha 3: (x - 1)/(x + 1)
				\node at (-6,1.1) {\small $\dfrac{x - 1}{x + 1}$};
				\draw[->] (-5,1.1) -- (5.5,1.1);
				\foreach \x/\lab/\cor in {-3/+/blue, 0/-/red, 2/+/blue, 4/+/blue}
				\node[below, text=\cor] at (\x,1.1) {\lab};
				\draw[fill=black] (-1,1.1) circle (2pt);
				\draw[fill=black] (1,1.1) circle (2pt);
				
				% Pontos verticais: -1 e 1
				\draw[thick, dashed] (-1,1.0) -- (-1,4.1); 
				\node[below] at (-1,1.0) {\scriptsize $-1$};
				
				\draw[thick, dashed] (1,1.0) -- (1,4.1);   
				\node[below] at (1,1.0) {\scriptsize $1$};
				
			\end{tikzpicture}
		\end{center}
		
		\begin{align*}
			\text{Solução:} \quad &x \in \,]- \infty,-1] \cup [1,+ \infty[ \\
			\text{Domínio:} \quad &]- \infty,-1] \cup [1,+ \infty[
		\end{align*}
		
	\end{practice}
	
	\pagebreak
	\section{Encontro 3 - 11 de julho de 2025}
	\subsection{Módulo}
	
	O módulo de um número $x$, ou o valor absoluto de $x$, é graficamente representado como a distância entre $x$ e $0$ na reta dos números reais. Representamos o módulo de $x$ por $\left| x \right|$.
	
	\begin{figure}[h]
		\centering
		\begin{tikzpicture}[scale=1.2]
			
			% Reta dos reais
			\draw[->] (-4.5,0) -- (4.5,0) node[right] {\small $\mathbb{R}$};
			
			% Zero
			\draw[fill=black] (0,0) circle (2pt);
			\node[below] at (0,-0.1) {\small $0$};
			
			% Ponto positivo x = 3
			\draw[fill=blue!60] (3,0) circle (2pt);
			\node[below] at (3,-0.1) {\small $x = 3$};
			
			% Ponto negativo x = -2
			\draw[fill=red!60] (-2,0) circle (2pt);
			\node[below] at (-2,-0.1) {\small $x = -2$};
			
			% Segmento de distância |3|
			\draw[<->, thick, blue] (0,0.5) -- (3,0.5);
			\node[above, text=blue] at (1.5,0.5) {\footnotesize $\left|3\right| = 3$};
			
			% Segmento de distância |-2|
			\draw[<->, thick, red] (0,1) -- (-2,1);
			\node[above, text=red] at (-1,1) {\footnotesize $\left|-2\right| = 2$};
			
		\end{tikzpicture}
		\caption{A distância de um número $x$ até o zero na reta representa o valor absoluto de $x$, isto é, $|x|$.}
		\label{fig:modulo-como-distancia}
	\end{figure}
	
	
	Assim, podemos definir módulo da seguinte forma:
	
	\begin{definition}[Módulo de um número $x$]
		\begin{align*}
			\left| x \right| = \begin{cases}
				x & \text{, se } x \ge 0\\
				-x &  \text{, se } x < 0
			\end{cases}
		\end{align*}
	\end{definition}
	
	\begin{example}Calcule $|x - 1|$. A partir da definição, temos
		
		\begin{align*}
			|x - 1| =
			\begin{cases}
				x - 1, & \text{se } x - 1 \geq 0 \\
				- (x - 1), & \text{se } x - 1 < 0
			\end{cases}
		\end{align*}
		
		Para cada caso, analisamos os sinais:
		\begin{align*}
			x - 1 \geq 0 &\Rightarrow x \geq 1 \\
			x - 1 < 0 & \Rightarrow x < 1
		\end{align*}
		
		Assim, temos
		\begin{align*}
			|x - 1| =
			\begin{cases}
				x - 1, & \text{se } x \geq 1 \\
				- (x - 1), & \text{se }  x < 1
			\end{cases}
		\end{align*}
	\end{example}
	
	\begin{example} Calcule $|(x-2)(x-4)|$.
		
		Para isso, é necessário analisar o sinal de cada expressão.
		\begin{align*}
			|(x-2)(x-4)| =
			\begin{cases}
				{\textcolor{olive}{(x-2)}}{\textcolor{red}{(x-4)}}, & \text{se } (x-2)(x-4) \geq 0 \implies x \geq ?\\
				- {\textcolor{olive}{(x-2)}}{\textcolor{red}{(x-4)}}, & \text{se }  (x-2)(x-4) < 0 \implies x < ?
			\end{cases}
		\end{align*}
		
		Vejamos o primeiro caso:
		
		\begin{align*}
			{\textcolor{olive}{(x-2)}}{\textcolor{red}{(x-4)}} &\geq 0\\
			\text{ou } {\textcolor{olive}{(x-2)}} \geq 0 \quad \text{, ou }&  {\textcolor{red}{(x-4)}} \geq 0
		\end{align*}
		
		Assim,
		\begin{align*}
			\text{(Primeira parte) }{\textcolor{olive}{x-2}}&\geq 0\\
			{\textcolor{olive}{x}} &\geq 2\\
			\text{(Segunda parte) } {\textcolor{red}{x-4}} &\geq 0\\
			{\textcolor{red}{x}} &\geq 4\\
		\end{align*}
		
		\begin{center}
			\begin{tikzpicture}[scale=1.1]
				
				% Linha 1: x - 2
				\node at (-6,3.9) {\small $\textcolor{olive}{x - 2}$};
				\draw[->] (-5,3.9) -- (6,3.9);
				\foreach \x/\lab/\cor in {-4/-/red, 0/-/red, 1/-/red, 3/+/blue, 5/+/blue}
				\node[below, text=\cor] at (\x,3.9) {\lab};
				
				% Linha 2: x - 4
				\node at (-6,2.5) {\small $\textcolor{red}{x - 4}$};
				\draw[->] (-5,2.5) -- (6,2.5);
				\foreach \x/\lab/\cor in {-4/-/red, 0/-/red, 1/-/red, 3/-/red, 5/+/blue}
				\node[below, text=\cor] at (\x,2.5) {\lab};
				
				% Linha 3: (x - 2)(x - 4)
				\node at (-6,1.1) {\small $\textcolor{olive}{(x - 2)}\textcolor{red}{(x - 4)}$};
				\draw[->] (-5,1.1) -- (6,1.1);
				
				% Sinais nos intervalos
				\foreach \x/\lab/\cor in {-4/+/blue, 0/+/blue, 1/+/blue, 3/-/red, 5/+/blue}
				\node[below, text=\cor] at (\x,1.1) {\lab};
				
				% Pontos verticais: x = 2 e x = 4
				\draw[thick, dashed] (2,1.0) -- (2,4.1); 
				\node[below] at (2,1.0) {\scriptsize $2$};
				
				\draw[thick, dashed] (4,1.0) -- (4,4.1);   
				\node[below] at (4,1.0) {\scriptsize $4$};
				
				% Bolinha fechada em x = 2
				\draw[fill=black] (2,1.1) circle (3pt);
				
				% Bolinha fechada em x = 4
				\draw[fill=black] (4,1.1) circle (3pt);
				
			\end{tikzpicture}
		\end{center}
		
		Observe que obtivemos os intervalos onde ${\textcolor{olive}{(x-2)}}{\textcolor{red}{(x-4)}} \geq 0$ e ${\textcolor{olive}{(x-2)}}{\textcolor{red}{(x-4)}} < 0$. Podemos completar nossa função:
		
		\begin{align*}
			|(x-2)(x-4)| &=
			\begin{cases}
				{\textcolor{olive}{(x-2)}}{\textcolor{red}{(x-4)}}, & \text{se } (x-2)(x-4) \geq 0 \implies x \leq 2 \text{ ou } x \geq 4\\
				- {\textcolor{olive}{(x-2)}}{\textcolor{red}{(x-4)}}, & \text{se }  (x-2)(x-4) < 0 \implies 2 < x < 4
			\end{cases}\\
			& \text{ou, simplesmente}\\
			|(x-2)(x-4)| &=
			\begin{cases}
				{\textcolor{olive}{(x-2)}}{\textcolor{red}{(x-4)}}, & \text{se } x \leq 2 \text{ ou } x \geq 4\\
				- {\textcolor{olive}{(x-2)}}{\textcolor{red}{(x-4)}}, & \text{se }   2 < x < 4
			\end{cases}
		\end{align*}
	\end{example}
	
	\textbf{Uma comparação visual entre a função $f(x) = x^2-6x+8$ e $|(x-2)(x-4)|$.}
	
	Experimente calcular $(x-2)(x-4)$. Fazendo a distributiva, teremos a mesma expressão $x^2-6x+8$ do Exemplo~\ref{ex:f_x2-6x+8} da página~\pageref{ex:f_x2-6x+8}.
	
	\begin{align*}
		(x-2)(x-4) &= x^2 - 4x - 2x + 8\\
		&=x^2 -6x + 8
	\end{align*}
	
	Graficamente, o que o módulo faz é refletir os valores negativos de $y$, como um espelho. Observe o comportamento das duas funções na Fig.~\ref{fig:modulo}.
	
	\begin{figure}[h]
		\centering
		
		% Painel 1: função f(x) = x^2 - 6x + 8
		\begin{subfigure}[t]{0.48\textwidth}
			\centering
			\begin{tikzpicture}
				\begin{axis}[
					title={$\displaystyle f(x) = x^2 - 6x + 8$},
					axis lines=middle,
					xmin=0, xmax=6,
					ymin=-2, ymax=10,
					samples=100,
					grid=both,
					xtick={2,4},
					ytick={0,8},
					enlargelimits=true,
					]
					\addplot[domain=0:6, thick, blue] {x^2 - 6*x + 8};
					\addlegendentry{Função quadrática}
				\end{axis}
			\end{tikzpicture}
			\caption{Gráfico da função original \( f(x) = x^2 - 6x + 8 \).}
			\label{fig:quadratica}
		\end{subfigure}
		\hfill
		% Painel 2: função com módulo |(x-2)(x-4)|
		\begin{subfigure}[t]{0.48\textwidth}
			\centering
			\begin{tikzpicture}
				\begin{axis}[
					title={$\displaystyle g(x) = |(x - 2)(x - 4)|$},
					axis lines=middle,
					xmin=0, xmax=6,
					ymin=0, ymax=10,
					samples=500,
					grid=both,
					xtick={2,4},
					ytick={0,8},
					enlargelimits=true,
					]
					\addplot[domain=0:2, thick, red] {(x - 2)*(x - 4)};
					\addplot[domain=2:4, thick, red] {-(x - 2)*(x - 4)};
					\addplot[domain=4:6, thick, red] {(x - 2)*(x - 4)};
					\addlegendentry{Função módulo}
				\end{axis}
			\end{tikzpicture}
			\caption{Gráfico de \( g(x) = |(x-2)(x-4)| \), com reflexões dos valores negativos.}
			\label{fig:modulo}
		\end{subfigure}
		
		\caption{Comparação entre a função quadrática \( f(x) = x^2 - 6x + 8 \) e a função módulo \( |(x-2)(x-4)| \). Observe que o gráfico da função com módulo reflete os valores negativos da parábola para cima, gerando uma curva sempre não negativa.}
		\label{fig:comparacao-funcao-modulo}
	\end{figure}
	
	
	\subsection{Funções compostas}
	
	A composição de funções é uma operação entre duas ou mais funções, em que a saída de uma função é a entrada da outra função.
	
	\begin{definition}[Função composta]
		Sejam três conjuntos $A,\, B,\, C$, e sejam duas funções $f: A \longrightarrow B$ e $g: B \longrightarrow C$. Dizemos que a função $g \circ f$ é uma função de $A$ em $C$.
		
		Denotamos essa função por $(g \circ f)(x) = g(f(x))$, em que $x \in A$. Lemos \textit{``$g$ composta $f$''}, ou \textit{``$g$ composta com $f$''}. 
		
		Observe, também, que $Im_f \subseteq D_g$ (lê-se \textit{``a imagem de $f$ é subconjunto do domínio de $g$''}, ou ainda \textit{``todos os elementos que se encontram na imagem de $f$ também se encontram no domínio de $g$''}).
	\end{definition}

	
	Graficamente, podemos representar $(g \circ f)(x) = g(f(x))$ da seguinte forma:
	
	\begin{center}
		\begin{tikzpicture}[scale=1.7]
			
			% Conjunto A
			\draw[fill=blue!10] (-10,0) rectangle (-8,4);
			\node at (-9,-0.5) {$D_f$};
			
			% Conjunto B
			\draw[fill=orange!10] (-6,0) rectangle (-4,4);
			\node at (-5,-0.5) {$D_g$};
			
			% Conjunto Im_f
			\draw[fill=blue!10] (-5.5,0.5) rectangle (-4.5,3.5);
			\node at (-5,0.7) {$Im_f$};
			
			% Conjunto C
			\draw[fill=green!10] (-2,0) rectangle (0,4);
			\node at (-1,-0.5) {$C = \text{contradomínio da função } g$};
			
			
			% Conjunto Im_g
			\draw[fill=orange!10] (-1.5,0.5) rectangle (-0.5,3.5);
			\node at (-1,0.7) {$Im_g$};
			
			% Conjunto Im_{f \circ g}
			\draw[fill=blue!10] (-1.3,1) rectangle (-0.7,3);
			\node at (-1,1.2) {\scriptsize $Im_{g \circ f}$};
			
			% Elementos
			\node[fill=black, circle, inner sep=2pt] (a) at (-9,2) {};
			\node[above right] at (-9,2) {\small $\mathbf{x}$};
			
			\node[fill=black, circle, inner sep=2pt] (b) at (-5,2) {};
			\node[above right] at (-5,2) {\small {$\mathbf{f(x)}$}};
			
			\node[fill=black, circle, inner sep=2pt] (c) at (-1,2) {};
			\node[above right] at (-1,2) {\small {$\mathbf{g\left(f(x)\right)}$}};
			
			% Setas
			% x -> f(x)
			\draw[->, thick] (-8.9,2) ..controls (-7.95,2.3) and (-6.05,2.3) .. (-5.1,2) node[midway, above] {\small $f$};
			
			% f(x) -> g(f(x))
			\draw[->, thick] (-4.9,2) ..controls (-3.95, 2.3) and (-2.05, 2.3) .. (-1.1,2) node[midway, above] {\small $g$};
			
			% g(f(x))
			\draw[->, thick] (a) .. controls (-9,5) and (-1,5) .. (c) node[midway, above] {\small $g \circ f$};
			
			% Transformações
			\draw[dashed] (-8,4) -- (-0.7,3); % Conjunto A
			\draw[dashed] (-8,0) -- (-0.7,1);
			
			\draw[dashed] (-4,4) -- (-0.5,3.5); % Conjunto B
			\draw[dashed] (-4,0) -- (-0.5,0.5);
			
			% Rótulo
			\node at (-5,-1) {Diagrama da função $g \circ f$.};
			
		\end{tikzpicture}
	\end{center}

	\begin{example}\label{ex:funcoes_compostas} Sejam duas funções $f(x) = \sqrt{x}$ e $g(x) = x + 1$. A função $f$ composta com a função $g$, que denotamos por $(f \circ g)(x)$ ou $f(g(x))$ é $f(g(x)) = \sqrt{g(x)} = \sqrt{x+1}$.
	
	Analogamente, a função $g$ composta com a $f$ denotamos $(g \circ f)(x) = g(f(x)) = f(x) + 1 = \sqrt{x} + 1$.
	
	\end{example}	
	
	\textbf{Analisando o domínio das funções compostas.}
	
	Observe que as funções do Exemplo~\ref{ex:funcoes_compostas} possuem domínios diferentes, e que $Im_g$, representada pelo retângulo amarelo, não está no domínio da função $f$. Podemos restringir o domínio de $g$ para que $Im_g \subseteq D_f$.
	
	\begin{align*}
		f(x) &= \sqrt{x} \implies x \geq 0\\
		D_f = [0, + \infty[,& \quad Im_f = [0, + \infty[\\
		\\
		g(x) &= x + 1\\
		D_g = \mathbb{R},& \quad Im_g = \mathbb{R}\\
		\\
		f(g(x)) &= \sqrt{x + 1} \implies x + 1 \geq 0 \implies x \geq -1\\
		\text{Domínio de } f \circ g = [-1, + \infty[,& \quad \text{Imagem de } f \circ g = [0, + \infty[\\
		\\
		g(f(x)) &= \sqrt{x} + 1 \implies x \geq 0\\
		\text{Domínio de } g \circ f = [0, + \infty[,& \quad \text{Imagem de } g \circ f = [1, + \infty[
	\end{align*}
	
	\begin{figure}[h]
		\centering
		% Painel 1: f(g(x)) = sqrt(x + 1)
		\begin{subfigure}[t]{0.48\textwidth}
			\centering
			\begin{tikzpicture}
				\begin{axis}[
					title={$\displaystyle f(g(x)) = \sqrt{x + 1}$},
					axis lines=middle,
					xmin=-1, xmax=5,
					ymin=0, ymax=4,
					samples=200,
					grid=both,
					xtick={-1,0,1,2,3,4,5},
					ytick={1,2,3,4},
					enlargelimits=true,
					]
					\addplot[domain=-1:5, thick, blue] {sqrt(x + 1)};
				\end{axis}
			\end{tikzpicture}
			\caption{Gráfico da função composta \( f(g(x)) = \sqrt{x + 1} \). Começa em \( x = -1 \) com imagem em \( [0, +\infty[ \).}
			\label{fig:fgx}
		\end{subfigure}
		\hfill
		% Painel 2: g(f(x)) = sqrt(x) + 1
		\begin{subfigure}[t]{0.48\textwidth}
			\centering
			\begin{tikzpicture}
				\begin{axis}[
					title={$\displaystyle g(f(x)) = \sqrt{x} + 1$},
					axis lines=middle,
					xmin=-1, xmax=5,
					ymin=0, ymax=4,
					samples=200,
					grid=both,
					xtick={-1,0,1,2,3,4,5},
					ytick={1,2,3,4},
					enlargelimits=true,
					]
					\addplot[domain=-1:5, thick, red] {sqrt(x) + 1};
				\end{axis}
			\end{tikzpicture}
			\caption{Gráfico da função composta \( g(f(x)) = \sqrt{x} + 1 \), definida em \( x \geq 0 \), com imagem em \( [1, +\infty[ \).}
			\label{fig:gfx}
		\end{subfigure}
		
		\caption{Comparação entre as funções compostas \( f(g(x)) = \sqrt{x + 1} \) e \( g(f(x)) = \sqrt{x} + 1 \). Embora ambas sejam crescentes e indefinidas para certos valores, \( f(g(x)) \) começa em 0 e \( g(f(x)) \) começa em 1.}
		\label{fig:comparacao-funcao-composta}
	\end{figure}
	
	
	\subsection{Revisão de conceitos de geometria analítica}
	
	\textbf{A distância entre dois pontos no plano cartesiano.}
	
	\begin{snugshade}
		\textbf{\textcolor{blue}{Relembre!}}
		
		O plano cartesiano é um conjunto de pares ordenados, e os pares ordenados nos dão os endereços desses pontos.
		
	\end{snugshade}
	
	Considere os pontos $A = (x_1, y_1)$ e $B = (x_2, y_2)$ no plano cartesiano abaixo. Qual a distância entre os pontos $A$ e $B$ indicada pela reta $D$?
	
	\begin{figure}[h]
		\centering
		\begin{tikzpicture}[scale=2]
			% Eixos
			\draw[->] (-0.5,0) -- (4,0) node[right] {\scriptsize $x$};
			\draw[->] (0,-0.5) -- (0,4) node[above] {\scriptsize $y$};
			
			% Pontos
			\coordinate (A) at (1,1);
			\coordinate (B) at (3,3);
			
			% Marcar pontos
			\fill (A) circle (1.5pt) node[below left] {\scriptsize $A = (x_1, y_1)$};
			\fill (B) circle (1.5pt) node[above right] {\scriptsize $B = (x_2, y_2)$};
			
			% Linha entre A e B
			\draw[thick] (A) -- (B) node[midway, above left] {\scriptsize $D$};
			
			% Projeções e retângulo auxiliar
			\draw[dashed] (B) -- ++(-3,0) node[left] {\scriptsize $y_2$};
			\draw[dashed] (B) -- ++(0,-3) node[below] {\scriptsize $x_2$};
			
			\draw[dashed] (A) -- ++(-1,0) node[left] {\scriptsize $y_1$};;
			\draw[dashed] (A) -- ++(0,-1) node[below] {\scriptsize $x_1$};;
			
			\draw[blue, thick] (A) -- ++(2,0) node[midway, below] {\scriptsize $x_2 - x_1$};
			\draw[red, thick] (B) -- ++(0,-2) node[midway, right] {\scriptsize $y_2 - y_1$};
			
			% Ângulo θ
			\draw (A) ++(0.5,0) arc[start angle=0,end angle=45,radius=0.5];
			\node at (1.6,1.3) {$\theta$};
			
			% Símbolo de ângulo reto
			\draw (3,1) -- ++(-0.2,0) -- ++(0,0.2) -- ++(0.2,0);
		\end{tikzpicture}
		\caption{Calculando a distância entre dois pontos no plano.}
		\label{fig:distancia-vetorial}
	\end{figure}

	Podemos calcular a distância $D$ usando o teorema de Pitágoras.
	
	\begin{proof}[Cálculo da distância entre dois pontos em um plano.]
		\begin{align*}
			D^2 &= (x_2 - x_1)^2 + (y_2 - y_1)^2  \quad \text{(aplicando o teorema de Pitágoras)}\\
			\sqrt[{\cancel{2}}]{D^{\cancel{2}}} &= \pm \sqrt{(x_2 - x_1)^2 + (y_2 - y_1)^2} \quad \text{(Extraímos a raiz de ambos os lados)}\\
			D &= \sqrt{(x_2 - x_1)^2 + (y_2 - y_1)^2} \qedhere
		\end{align*}
	
		É comum denotarmos a distância $D$ por $d(A,B)$ nos livros. Assim, vamos usar $d(A,B)$ quando nos referirmos a distância entre dois pontos, da seguinte forma:
		
		\begin{equation}
			d(A,B) = \sqrt{(x_2 - x_1)^2 + (y_2 - y_1)^2}
		\end{equation}
		
	\end{proof}

	\begin{snugshade}
		\textbf{\textcolor{red}{Importante!}}
	
		Perceba que a distância é sempre maior ou igual a zero. Não faria sentido termos uma distância negativa entre dois pontos, no nosso contexto.
	\end{snugshade}
	
	\textbf{Calculando a inclinação da reta}
	
	Observe, a partir da Fig.~\ref{fig:distancia-vetorial}, que se formou um ângulo $\theta$ entre os lados $(x_2 - x_1)$ e a reta $D$. Já sabemos calcular essa inclinação (veja Pág.~\pageref{tema:coef_angular}). Basta calcularmos a tangente do ângulo ($\tan{\theta}$). Podemos generalizar e calcular a inclinação para quaisquer pontos $x$ e $y$, da seguinte forma:
	
	\begin{proof}[A equação da reta.]
		\begin{align*}
			\tan{\theta} &= \frac{\text{cateto oposto}}{\text{cateto adjacente}}\\
			\tan{\theta} &= \frac{y - y_0}{x - x_0}\\
			\tan{\theta} &= \frac{y - y_0}{x - x_0}=m \quad \text{(onde $m$ é o coeficiente angular)}
		\end{align*}
		
		Podemos reorganizar a equação para a forma que geralmente encontramos nos livros de matemática:
		
		\begin{align*}
			\frac{y - y_0}{x - x_0} &= m\\
			y - y_0 &= m(x - x_0) \qedhere
		\end{align*}
	\end{proof}
	
	\textbf{A distância entre qualquer ponto no plano e o centro de uma circunferência.}
	
	Agora, e se quisermos calcular a distância entre o centro e um ponto $P$ sobre a linha de uma circunferência?
	
	\begin{figure}[h]
		\centering
		\begin{tikzpicture}[scale=2]
			% Eixos
			\draw[->] (-0.5,0) -- (3,0) node[right] {\scriptsize $x$};
			\draw[->] (0,-0.5) -- (0,3) node[above] {\scriptsize $y$};
			
			% Centro e raio
			\coordinate (C) at (2,2);
			\def\radius{1}
			
			% Ponto na circunferência
			\coordinate (P) at ($(C)+(45:\radius)$); % 45° a partir do centro
			
			% Circunferência
			\draw[thick, blue] (C) circle (\radius);
			
			% Reta R entre C e P
			\draw[red, thick] (C) -- (P) node[midway, left] {\scriptsize $R$};
			
			% Marcar os pontos
			\fill (C) circle (1pt) node[below left] {\scriptsize $C$};
			\fill (P) circle (1pt) node[above right] {\scriptsize $P = (x,y)$};
			
			% Projeções nos eixos
			\draw[dashed] (C) -- ++(-2,0) node[left] {\scriptsize $y_c$}; 
			\draw[dashed] (C) -- ++(0,-2) node[below] {\scriptsize $x_c$};
		\end{tikzpicture}
		\caption{Circunferência com centro \( C \) e ponto \( P \) sobre sua borda. A reta \( R \) representa o raio entre \( C \) e \( P \).}
		\label{fig:circunferencia-com-raio}
	\end{figure}
	
	\begin{snugshade}
		\textbf{\textcolor{red}{Importante!}}
		
		Algumas coisas a considerar sobre a Fig.~\ref{fig:circunferencia-com-raio}:
		\begin{enumerate}
			\item O ponto $P$ está sobre o contorno da circunferência, destacada em \textcolor{blue}{azul};
			\item A medida entre o centro e o contorno da circunferência tem raio $R$;
			\item O valor do raio $R$ é o mesmo do centro a qualquer ponto do contorno;
		\end{enumerate}
		
		Isso significa que podemos calcular a distância entre o centro $C$ e qualquer ponto $P$ que se encontre no contorno da circunferência. Isto é, podemos calcular para qualquer $P = (x,y)$.
	\end{snugshade}
	
	\begin{proof}[A distância entre o centro e um ponto $P$ qualquer ]
		\begin{align*}
			d(P,C) = R &= \sqrt{(y - y_1)^2 + (x - x_1)^2} \\
			R^2 &= \left(\sqrt{(y - y_1)^2 + (x - x_1)^2}\right) ^2 \quad \text{(Eleve os dois lados ao quadrado para simplificar a raiz)}\\
			R^2 &= (y - y_1)^2 + (x - x_1)^2 \qedhere
		\end{align*}
	\end{proof}
	
	\begin{example} Seja $C = (1,2)$ o centro de uma circunferência de raio $R = 2$.
		A equação dessa circunferência é
		\begin{align*}
			R^2 &= (x - x_1)^2 +(y - y_1)^2 \\
			2^2 &= (x - x_1)^2 +(y - 2)^2 \quad \text{(Substitua $x_1=1$ e $y_1 = 2$)}\\
			4 &= (x - x_1)^2 +(y - 2)^2
		\end{align*}
	\end{example}
	
	Com essa expressão podemos desenhar o círculo da Fig.~\ref{fig:circunferencia-la}.
	
	\begin{figure}[h]
		\centering
		\begin{tikzpicture}[scale=1.5]
			% Eixos
			\draw[->] (-1,0) -- (4,0) node[right] {\scriptsize $x$};
			\draw[->] (0,-1) -- (0,4) node[above] {\scriptsize $y$};
			
			% Centro da circunferência
			\coordinate (C) at (1,2);
			
			% Raio
			\def\radius{2}
			
			% Circunferência
			\draw[thick, blue] (C) circle (\radius);
			
			% Marcar centro e ponto na borda
			\fill (C) circle (1pt) node[below left] {\scriptsize $(1, 2)$};
			
			% Ponto na circunferência
			\coordinate (P) at ($(C)+(0:\radius)$);
			\fill (P) circle (1pt) node[right] {\scriptsize $P$};
			
			% Raio
			\draw[red, thick] (C) -- (P) node[midway, below] {\scriptsize $r = 2$};
		\end{tikzpicture}
		\caption{A curva em azul representa a circunferência dada por \( 4 = (y - 2)^2 + (x - 1)^2 \), com centro em \( (1, 2) \) e raio \( r = 2 \).}
		\label{fig:circunferencia-la}
	\end{figure}
	
	\begin{snugshade}
		\textbf{\textcolor{red}{Importante!}}
		
		A equação $R^2 = (x - x_1)^2 + (y - y_1)^2$ é uma função? \textbf{Não!}
		
		Lembre-se de que uma condição necessária para termos uma função é que, para cada $x$ em um conjunto $A$, exista apenas um $y$ em um conjunto $B$. Chamamos isso de ``critério de unicidade''. Contudo, em um círculo encontramos um mesmo $x$ relacionado a mais de um $y$. Isso fica mais evidente ao traçarmos retas paralelas ao eixo $x$.
		
		\begin{center}
			\begin{tikzpicture}[scale=1]
				% Eixos
				\draw[->] (-3,0) -- (3,0) node[right] {\( x \)};
				\draw[->] (0,-2.5) -- (0,2.5) node[above] {\( y \)};
				
				% Circunferência
				\draw[thick,blue] (0,0) circle (2);
				
				% Retas verticais
				\foreach \x in {-1.5,-1,0,1,1.5} {
					\draw[dashed,red] (\x,-2.5) -- (\x,2.5);
				}
				
				% Rótulo
				\node[align=center] at (0,-3) {Não é uma função! Observe como um $x$ possui mais de um $y$ relacionado, ferindo \\a condição de unicidade.};
			\end{tikzpicture}
		\end{center}	
		
	\end{snugshade}
	
	\subsection{Revisão de outros conceitos de trigonometria}
	
	O que é um radiano ($rad$)? Considere a circunferência abaixo.
	
	\begin{center}
		\begin{tikzpicture}[scale=1.5]
			% Raio
			\def\R{2}        % Raio da circunferência
			\def\Angle{57.3}   % ~1 rad em graus

			
			% Centro da circunferência
			\coordinate (O) at (0,0);
			\fill (O) circle (0.5pt) node[below left] {\scriptsize Centro};
			
			% Circunferência completa em cinza claro
			\draw[thick] (O) circle (\R);
			
			% Raio principal em azul
			\draw[blue, thick] (O) -- (\Angle:\R) node[midway, left] {\scriptsize $R$};
			
			% Segundo raio (linha horizontal) para formar o ângulo
			\draw[blue, thick] (O) -- (\R,0) node[midway, below] {\scriptsize $R$};
			
			% Arco correspondente ao ângulo θ
			\draw[violet, thick] (0.4,0) arc[start angle=0, end angle=\Angle, radius=0.4];
			
			% Indicação do ângulo θ
			\node[violet] at (0.75,0.3) {\scriptsize $\theta = 1\,\text{rad}$};
			
			% Arco da circunferência (R)
			\draw[red, thick] (2,0) arc[start angle=0, end angle=\Angle, radius=2];
			\node[red] at (2, 1.1) {\scriptsize $R$};
			
		\end{tikzpicture}
	\end{center}
	
	\begin{definition}[Radiano]
		\textcolor{violet}{Um radiano} é a medida do ângulo central que gera um \textcolor{red}{arco de circunferência ($R$)} com comprimento igual ao \textcolor{blue}{raio da circunferência ($R$)}.
	\end{definition}
	
	Trabalhamos com números em radianos porque eles são muito convenientes e facilitam cálculos.
	
	\textbf{Quantos radianos há em uma circunferência?}
	
	Imagine uma linha com comprimento $R$, e tente dispô-la sobre o contorno da circunferência. Você perceberá que caberão três linhas com comprimento $R$ e ainda faltará um pedaço para cobrir metade da circunferência.
	
			\begin{center}
		\begin{tikzpicture}[scale=1.5]
			% Raio
			\def\R{2}        % Raio da circunferência
			\def\Angle{57.3}   % ~1 rad em graus
			
			
			% Centro da circunferência
			\coordinate (O) at (0,0);
			\fill (O) circle (0.5pt) node[below left] {\scriptsize Centro};
			
			% Circunferência completa em cinza claro
			\draw[thick] (O) circle (\R);
			
			% Raio principal em azul
			\draw[blue, thick] (O) -- (3*\Angle:\R) node[midway, below left] {\scriptsize $R$};
			
			% Segundo raio (linha horizontal) para formar o ângulo
			\draw[blue, thick] (O) -- (\R,0) node[midway, below] {\scriptsize $R$};
			
			% Arco correspondente ao ângulo θ
			\draw[violet, thick] (0.4,0) arc[start angle=0, end angle=3*\Angle, radius=0.4];
			
			% Indicação do ângulo θ
			\node[violet] at (0.2,0.55) {\scriptsize $\theta = 3\,\text{ rad}$};
			
			% Arco da circunferência (R)
			\draw[red, thick] (2,0) arc[start angle=0, end angle=3*\Angle, radius=2];
			\node[red] at (0, 2.2) {\scriptsize $R$};
		
		\end{tikzpicture}
	\end{center}
	
	O comprimento do pedaço que falta é algo muito próximo de $R \cdot 0,141592...$ Isso acontece porque o número de arcos com o mesmo tamanho do raio que completam essa metade da circunferência é um número irracional, que chamamos de $\pi$ (lê-se \textit{``pi''}).
	
	\begin{equation*}
		\pi \approx 3,141592653 ...\text{( lê-se \textit{``pi é aproximadamente três vírgula quatorze...''})}
	\end{equation*}
	
	Ou seja, em metade de uma circunferência há $\pi\, rad$. Em uma circunferência completa há $2 \pi\, rad$.
	
	Observe que para calcularmos quantos arcos cabiam na circunferência contamos $R \cdot \pi$ arcos. É daí que vem a fórmula do comprimento da circunferência:
	
	\begin{equation*}
		C = 2 \pi R
	\end{equation*}
	
	\begin{condicoes}
		C & comprimento da circunferência completa\\
		2 \pi & em radianos, o ângulo completo da circunferência\\
		R & o raio da circunferência
	\end{condicoes}
	
	Para calcularmos qualquer comprimento de arco, calculamos
	
	\begin{equation*}
		S = \theta \cdot R
	\end{equation*}
	
	\begin{condicoes}
		S & comprimento do arco\\
		\theta & ângulo, em radianos\\
		R & raio da circunferência
	\end{condicoes}

	
	\textbf{Funções trigonométricas.}
	
	Considere o triângulo retângulo a seguir. Um triângulo retângulo é aquele que possui um de seus ângulos igual a $90^\circ$.
	
	\begin{center}
		\begin{tikzpicture}[scale=1]
			% Coordenadas dos pontos do triângulo
			\coordinate (A) at (0,0); % Ângulo theta
			\coordinate (B) at (5,0); % Base
			\coordinate (C) at (5,3); % Altura
			
			% Triângulo
			\draw[thick] (A) -- (B) -- (C) -- cycle;
			
			% Lados
			\node[red] at (2.5,-0.2) {\small $b$};       % base
			\node[blue] at (5.2,1.5) {\small $a$};       % altura
			\node[violet] at (2.5,1.7) {\small $c$};     % hipotenusa
			
			% Marcação de ângulo reto
			\draw (B) ++(-0.3,0) -- ++(0,0.3) -- ++(0.3,0);
			
			% Indicação do ângulo θ entre lado b e hipotenusa
			\draw (A) ++(1,0) arc[start angle=0, end angle=45, radius=0.7071068];
			\node at (1.2,0.3) {\small $\theta$};
		\end{tikzpicture}
	\end{center}
	
	\begin{condicoes}
		\textcolor{red}{b} & cateto adjacente ao ângulo $\theta$\\
		\textcolor{blue}{a} & cateto oposto ao ângulo $\theta$\\
		\textcolor{violet}{c} & hipotenusa
	\end{condicoes}
	
	A partir do triângulo podemos encontrar as seguintes relações trigonométricas:
	
	\begin{equation}
		\sin{\theta} = \frac{\text{cateto oposto}}{\text{hipotenusa}} = \frac{a}{c}
	\end{equation}
	
	\begin{equation}
		\cos{\theta} = \frac{\text{cateto adjacente}}{\text{hipotenusa}} = \frac{b}{c}
	\end{equation}
	
	\begin{equation}
		\tan{\theta} = \frac{\text{cateto oposto}}{\text{cateto adjacente}} = \frac{a}{b}
	\end{equation}
	
	\begin{proof}[Mostrando que $\tan{\theta} = \displaystyle \frac{\sin{\theta}}{\cos{\theta}}$]
		
		\begin{align*}
			\frac{\sin{\theta}}{\cos{\theta}} &= \displaystyle \frac{\displaystyle \frac{a}{c}}{\displaystyle \frac{b}{c}}\\
			&= \displaystyle \frac{a}{c} \cdot \frac{c}{b} \quad \text{(Dividir por fração é multiplicar pelo inverso)}\\
			&= \displaystyle \frac{a}{{\cancel{c}}} \cdot \frac{{\cancel{c}}}{b} \quad \text{(Simplifique $\frac{c}{c} = 1$)}\\
			&= \frac{a}{b} = \tan{\theta} \qedhere
		\end{align*}
	\end{proof}
	
	A partir do triângulo retângulo da Pág.~\pageref{fig:triangulo_retangulo}, cujos lados são iguais, também podemos encontrar valores para alguns ângulos notáveis.
	
	\begin{example}Demonstre ângulos notáveis ($30^\circ, \, 45^\circ, \, 60^\circ$).
		
		Considerando que um triângulo retângulo tem um dos ângulos igual a $90^\circ$ graus, e que o triângulo todo possui $180^\circ$ graus, então:
		
		\begin{align*}
			\theta = \frac{180 - 90}{2} = 45^\circ = \frac{\pi}{4}
		\end{align*}
		
		\begin{center}\label{fig:triangulo_retangulo}
			\begin{tikzpicture}[scale=1]
				% Coordenadas dos pontos do triângulo
				\coordinate (A) at (0,0); % Ângulo theta
				\coordinate (B) at (3,0); % Base
				\coordinate (C) at (3,3); % Altura
				
				% Triângulo
				\draw[thick] (A) -- (B) -- (C) -- cycle;
				
				% Lados
				\node[blue] at (1.5,-0.2) {\small $l$};       % base
				\node[blue] at (3.2,1.5) {\small $l$};       % altura
				\node[violet] at (1.5,1.7) {\small $h$};     % hipotenusa
				
				% Marcação de ângulo reto
				\draw (B) ++(-0.3,0) -- ++(0,0.3) -- ++(0.3,0);
				
				% Indicação do ângulo θ entre lado b e hipotenusa
				\draw (A) ++(1,0) arc[start angle=0, end angle=45, radius=1];
				\node at (1.2,0.3) {\small $\theta$};
			\end{tikzpicture}
		\end{center}
		
		Assim, podemos dizer que
		
		\begin{align*}
			\tan{(\theta)} = \tan{\left(\frac{\pi}{4}\right)} = \frac{l}{l} = 1
		\end{align*}
		
		Também podemos encontrar que
		
		\begin{align*}
			h^2 &= l^2 + l^2 \quad \text{(Teorema de Pitágoras)}\\
			h^2 &= 2\cdot l^2\\
			h &= \sqrt{2\cdot l^2} \quad \text{(Tiramos a raiz dos dois lados)}\\
			h &= \sqrt{2} \cdot \sqrt{l^2} \quad \text{(Reescrevemos o produto de duas raízes)}\\
			h &= \sqrt{2} \cdot \sqrt[{\cancel{2}}]{l^{\cancel{2}}} \quad \text{(Simplificamos $\sqrt{l^2}$)}\\
			h &= l\sqrt{2}	
		\end{align*}
		
		E agora podemos encontrar
		
		\begin{align*}
			\sin{\theta} = \sin{\left(\frac{\pi}{4}\right)} &= \frac{l}{l\sqrt{2}}\\
			&= \frac{l}{l\sqrt{2}} \cdot \frac{\sqrt{2}}{\sqrt{2}} \quad \text{(Racionalizamos)}\\
			&= \frac{\cancel{l}}{\cancel{l}\sqrt{2}} \cdot \frac{\sqrt{2}}{\sqrt{2}} \quad \text{($\sqrt{2} \cdot \sqrt{2} = \sqrt{4} = 2$)}\\
			&= \frac{\sqrt{2}}{2}
		\end{align*}
		
		Observe que o mesmo serve para $\displaystyle \cos{\left(\frac{\pi}{4}\right)} = \frac{l}{l\sqrt{2}} = \frac{\sqrt{2}}{2}$.
		
	\end{example}
	
	\subsection{Resolvendo exercícios}
	
	\begin{practice}
		\textbf{Exercício 2 do livro.} Simplifique $\dfrac{f(x) - f(p)}{x - p}$, $(x \neq p)$ sendo dados:
		
		\begin{enumerate}
			\item[j)] $f(x) = \displaystyle \frac{1}{x}$ e $p$ qualquer
			
			\begin{align*}
				f(x) &= \dfrac{1}{x} \\
				f(p) &= \dfrac{1}{p} \\
				\\
				\dfrac{f(x) - f(p)}{x - p} &= \dfrac{\dfrac{1}{x} - \dfrac{1}{p}}{x - p} \\
				&= \dfrac{\dfrac{p - x}{xp}}{x - p} \quad \text{(Tire mmc ou use a multiplicação esperta $\dfrac{x}{x} = \dfrac{p}{p} = 1$)}\\
				&= \dfrac{-x + p}{xp}\cdot \frac{1}{(x - p)} \quad \text{(Divisão de fração: multiplique pelo inverso)}\\
				&= \dfrac{-1(x - p)}{xp(x - p)} \quad \text{(Repare: $-x + p = (-1)(x - p)$)}\\
				&= \dfrac{-1\cancel{(x - p)}}{xp \cancel{(x - p)}} \quad \text{(Simplificando $\dfrac{x-p}{x-p}$)}\\
				&= -\dfrac{1}{xp} \quad \text{(para } x \ne p\text{)}
			\end{align*}
			
		\end{enumerate}
	\end{practice}
	
	\pagebreak
	
	\section{Encontro 4 - 14 de julho de 2025}
	
	\subsection{Continuidade de funções}
	
	Uma função contínua é aquela cujo gráfico pode ser desenhado sem tirar o lápis do papel. 
	
		\textbf{Resumo de funções contínuas}
	
		Assim, as seguintes funções $f: \mathbb{R} \longrightarrow \mathbb{R}$ abaixo são contínuas:
		
		\begin{table}[h]
			\caption{Resumo de funções contínuas.}
			\centering
			\begin{tabular}[t]{|c|c|}
				\hline
				Função & Exemplos \\
				\hline
				Constantes & $f(x) = K, K \in \mathbb{R}$ \\
				\hline
				Polinomiais & $f(x) = mx + b$, $f(x) = ax^2 + bx + c$\\
				\hline
				Racionais & $f(x) = \dfrac{p(x)}{s(x)}$\\
				\hline
				Raízes & $f(x) = \sqrt[3]{x}$, $f(x) = \sqrt{x+1}$\\
				\hline
				Exponenciais & $f(x) = b^x$\\
				\hline
				Logarítmicas & $f(x) = \log x$\\
				\hline
				Trigonométricas & $f(x) = \sin x $, $f(x) = \cos x$\\
				\hline
			\end{tabular}
		\end{table}
	
	\subsection{Limites de funções}
	
	\begin{definition}[Limite de uma função]
		conteúdo...
	\end{definition}
	
	\begin{snugshade}
		\textbf{\textcolor{violet}{Saiba mais!}}
		
		Uma ferramenta útil para a fatoração é o Triângulo de Pascal, abaixo segue a forma de construí-lo e utilizá-lo. Também incluo algumas fatorações que poderão ser úteis ao longo do curso e na resolução de exercícios.
		
		\textbf{\large{Triângulo de Pascal}}
		
		O Triângulo de Pascal pode ser representado como um triângulo retângulo da seguinte forma:
		
		\[
		\begin{array}{cccccccc}
			1 &   &   &   &   &   &   & \\
			1 & 1 &   &   &   &   &   & \\
			1 & 2 & 1 &   &   &   &   & \\
			1 & 3 & 3 & 1 &   &   &   & \\
			1 & \textcolor{red}{4} & \textcolor{red}{6} & 4 & 1 &   &   & \\
			1 & 5 & \textcolor{red}{10} & 10 & 5 & 1 &   & \\
			1 & 6 & 15 & 20 & 15 & 6 & 1 & \\
			\vdots & \vdots & \vdots & \vdots & \vdots & \vdots & \vdots & \ddots \\
		\end{array}
		\]
		
		\textbf{Como construir o Triângulo de Pascal}
		
		\begin{itemize}
			\item A primeira coluna e a diagonal principal contêm apenas o número 1.
			\item Cada elemento interno é a soma do elemento acima dele com o elemento à esquerda do elemento acima. Note os números destacados em vermelho e que $4 + 6 = 10$
		\end{itemize}
		
		\textbf{\large{Conexão com Potências Binomiais}}
		
		Os coeficientes das expansões binomiais correspondem às linhas do Triângulo de Pascal:
		
		\textbf{Quadrado de um binômio}
		\[ (a + b)^2 = \boxed{1}a^2 + \boxed{2}ab + \boxed{1}b^2 \]
		Corresponde à $3$ª linha do triângulo: $1, 2, 1$.
		
		A expansão de $(a - b)^2$ seria:
		\[ (a - b)^2 = 1a^2 - 2ab + 1b^2 \]
		que usa os coeficientes da 3ª linha com sinais alternados.
		
		\textbf{Cubo de um binômio}
		\[ (a + b)^3 = \boxed{1}a^3 + \boxed{3}a^2b + \boxed{3}ab^2 + \boxed{1}b^3 \]
		Corresponde à $4$ª linha do triângulo: $1, 3, 3, 1$.
		
		A expansão de $(a - b)^3$ seria:
		\[ (a - b)^3 = a^3 - 3a^2b + 3ab^2 - b^3 \]
		que usa os coeficientes da 4ª linha com sinais alternados.
		
		\textbf{Quarta potência de um binômio}
		\[ (a + b)^4 = \boxed{1}a^4 + \boxed{4}a^3b + \boxed{6}a^2b^2 + \boxed{4}ab^3 + \boxed{1}b^4 \]
		Corresponde à $5$ª linha do triângulo: $1, 4, 6, 4, 1$.
		
		A expansão de $(a - b)^4$ seria:
		\[ (a - b)^4 = a^4 - 4a^3b + 6a^2b^2 - 4ab^3 + b^4 \]
		usando os coeficientes da $5$ª linha com sinais alternados.
		
		\textbf{\large{Outras fatorações importantes}}
		
		\textbf{Diferença de Quadrados}
		\[ a^2 - b^2 = (a - b)(a + b) \]
		
		\textbf{Diferença de Cubos}
		\[ a^3 - b^3 = (a - b)(a^2 + ab + b^2) \]
		
		\textbf{Quarta Diferença}
		\[ a^4 - b^4 = (a - b)(a^3 + a^2b + ab^2 + b^3) \]
		
		É importante notar que no lado direito da igualdade e nas expressões do segundo fator, a potência do $a$ começa em grau uma unidade menor, por exemplo, ser for uma diferença de termos elevados à quarta $(a^{4} - b^{4})$, teremos o inicío em $a^{3}$ e a potência do $b$ inicia em zero e, enquanto a potência do $a$ decresce, a do $b$ cresce, observe
		\[a^3b^{0} + a^2b^{1} + a^{1}b^2 + a^{0}b^3 = a^3 + a^2b + ab^2 + b^3\]
		
		Este raciocínio se aplica às demais diferenças entre potências de expoentes iguais.
	\end{snugshade}
	
	\subsection{Resolvendo exercícios}
	
	\pagebreak
	\section{Encontro 5 - 15 de julho de 2025}
	
	\subsection{Derivadas}
	
	\textbf{A derivada como um limite}
	
	\textbf{Regras de derivação}
	
	\begin{proof}[A derivada da soma de duas funções]
		conteúdo...
	\end{proof}
	
	\textbf{Tabela de derivação}
	
	\textbf{Lista de derivação}
	
	\textbf{Derivadas trigonométricas}
	
	\textbf{Revisando a tabela de derivação}
	
	\pagebreak
	\section{Encontro 6 - 16 de julho de 2025}
	
	\pagebreak
	\section{Encontro 7 - 17 de julho de 2025}
	
	\pagebreak
	\section{Encontro 8 - 18 de julho de 2025}
	
	\newpage
	\printbibliography
\end{document}

